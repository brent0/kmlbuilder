\nonstopmode{}
\documentclass[a4paper]{book}
\usepackage[times,inconsolata,hyper]{Rd}
\usepackage{makeidx}
\usepackage[utf8,latin1]{inputenc}
% \usepackage{graphicx} % @USE GRAPHICX@
\makeindex{}
\begin{document}
\chapter*{}
\begin{center}
{\textbf{\huge Package `kmlbuilder'}}
\par\bigskip{\large \today}
\end{center}
\begin{description}
\raggedright{}
\item[Type]\AsIs{Package}
\item[Title]\AsIs{Build KML Files from Data Frames}
\item[Depends]\AsIs{R.oo, rgdal, RCurl}
\item[Version]\AsIs{1.1.8}
\item[Date]\AsIs{2016-01-29}
\item[Author]\AsIs{Brent Cameron}
\item[Maintainer]\AsIs{Brent Cameron }\email{brentcameron1@gmail.com}\AsIs{}
\item[LazyData]\AsIs{true}
\item[Description]\AsIs{Easily generate keyhole markup files from dataframes. These files
can be used by Google Earth to view your geospatial data. Contains
functions that generate features for folders, points, polygons,
linestrings, image overlays, screen overlays and gives methods
to style these features.}
\item[ERRORS]\AsIs{Submit bug report at https://github.com/brent0/kmlbuilder}
\item[License]\AsIs{GPL-3}
\item[NeedsCompilation]\AsIs{no}
\end{description}
\Rdcontents{\R{} topics documented:}
\inputencoding{utf8}
\HeaderA{kmlbuilder-package}{Create Keyhole Markup Language from R}{kmlbuilder.Rdash.package}
\aliasA{kmlbuilder}{kmlbuilder-package}{kmlbuilder}
\keyword{package}{kmlbuilder-package}
%
\begin{Description}\relax
This package can be used to quickly create kml documents within R. You can easily create simple kml's with little or no styles, or, you can build much more complex kmls that have multiple features, folders and styles. This package supports most of the topics described in the kml reference document found here: https://developers.google.com/kml/documentation/kmlreference.   
\end{Description}
%
\begin{Details}\relax

\Tabular{ll}{
Package: & kmlbuilder\\{}
Type: & Package\\{}
Version: & 1.1.8\\{}
Date: & 2016-01-29\\{}
License: & GPL-3\\{}
}
Use this package if you would like to quickly generate kml documents from data stored in common R data structures
\end{Details}
%
\begin{Author}\relax
Brent Cameron\\{}

Maintainer: brentcameron1@gmail.com

\end{Author}
%
\begin{References}\relax
\Rhref{http://www.google.com/earth/download/ge/agree.html}{Google Earth Download}
\Rhref{https://developers.google.com/kml/documentation/kmlreference}{KML Reference}
\end{References}
%
\begin{Examples}
\begin{ExampleCode}
###  SIMPLE EXAMPLE  ###
mykml = RKmlObject() #Create kml object
lat = lat = rep(seq(44, 45, .1), 11) #latitude 
lon = rep(seq(-60, -61, -.1), 11) #longitude
lon = lon[order(lon)]
mykml$addPoint(x = cbind(lat, lon)) #add point features to kml
#mykml$preview() #sample the kml, used default styles

### END SIMPLE EXAMPLE  ###


### STYLE EXAMPLE  ###

#Create a RKmlObject
mykml = RKmlObject()

#Icon for use in styles
fishpath = "http://maps.google.com/mapfiles/ms/micons/fishing.png" 

#Create Styles
mykml$addLabelStyle(styleid = "my_style", color = "red", transparency = .5, scale = .5)
mykml$addIconStyle(styleid = "my_style", href = fishpath, color = "green", scale = 1, heading = 0)
mykml$addPolyStyle(styleid = "my_style", color = "yellow", transparency = .9, outline = 0, fill = 1)
mykml$addPolyStyle(styleid = "my_style2", color = "white", outline = 1, fill = 0)

#Create Folder Structure
mykml$addFolder(fid = "2013", name = "2013") #Demonstrate adding folder
mykml$getFolder("2013")$addFolder(fid = "survey_data", name = "survey data") #Demonstrate adding nested folder

# Assign Folder object to variable
sur2013 = mykml$getFolder("2013")$getFolder("survey_data")

lat = c(45.15, 44.95)
lon = c(-60.55, -60.15)
name = c("location1", "location2" )
TimeStamp = c("2013-09-20", "2013-10-22")

x = cbind(lat, lon, name, TimeStamp)

sur2013$addPoint(x, styleUrl = "my_style") #Demonstrate add data to folder

#one polygon inside another with different altitudes
pid = c(1, 1, 1, 1, 6, 6, 6, 6)
lat = c(45, 45.1, 45.1, 45, 45, 45.1, 45.1, 45)
lon = c(-60.5, -60.5, -60.2, -60.2, -60.5, -60.5, -60.2, -60.2)
x = cbind(pid, lat, lon)

#Define cut out polygon for polygon with pid = 1
pid = c(1, 1, 1)
lat = c(45.03, 45.06, 45.03)
lon = c(-60.3, -60.35, -60.4)
y = cbind(pid, lat, lon)

sty = c("my_style", "my_style2") #Cycles by pid
alt = c(7000, 10000) #Cycles by pid
mykml$addPolygon(x, y, styleUrl = sty, altitude = alt)

#mykml$preview()

### END STYLE EXAMPLE ### 
\end{ExampleCode}
\end{Examples}
\inputencoding{utf8}
\HeaderA{addAbstractView}{Add Abstract view to KML Object}{addAbstractView}
%
\begin{Description}\relax
After creating a kml object, this method defines an Abstract view that can be used by many elements of a kml. When a using an AbstractView, be sure that the AbstractView value equals the desired viewid argument.
\end{Description}
%
\begin{Usage}
\begin{verbatim}
yourKMLobj$addAbstractView(type = "camera", viewid, TimeStamp, ViewerOptions, longitude, latitude, altitude, heading, tilt, roll, altitudeMode) 
yourKMLobj$addAbstractView(type = "camera", viewid, TimeSpanStart, TimeSpanEnd, ViewerOptions, longitude, latitude, altitude, heading, tilt, roll, altitudeMode)

yourKMLobj$addAbstractView(type = "lookat", viewid, TimeStamp, ViewerOptions, longitude, latitude, altitude, heading, tilt, range, altitudeMode) 

yourKMLobj$addAbstractView(type = "lookat", viewid, TimeSpanStart, TimeSpanEnd, ViewerOptions, longitude, latitude, altitude, heading, tilt, range, altitudeMode) 
\end{verbatim}
\end{Usage}
%
\begin{Arguments}
\begin{ldescription}

\item[\code{type}] 
Mandatory. character. either 'lookat' or 'camera', the type of abstract view you would like to create.    

\item[\code{viewid}] 
Mandatory. Define the id for this styleview. kml elements will need to reference this id to use the defined view. 

\item[\code{TimeStamp}] character. Define a moment in time(formats: (YYYY)(YYYY-MM)(YYYY-MM-DD)(YYYY-MM-DDThh:mm:ssZ)(YYYY-MM-DDThh:mm:ss). This will affect the features that are displayed, sunlight and historical imagery.
\item[\code{TimeSpanStart}] character. Define the start of a span of time(formats: (YYYY)(YYYY-MM)(YYYY-MM-DD)(YYYY-MM-DDThh:mm:ssZ)(YYYY-MM-DDThh:mm:ss). This will affect the features that are displayed, sunlight and historical imagery.
\item[\code{TimeSpanEnd}] character. Define the end of a span of time(formats: (YYYY)(YYYY-MM)(YYYY-MM-DD)(YYYY-MM-DDThh:mm:ssZ)(YYYY-MM-DDThh:mm:ss). This will affect the features that are displayed, sunlight and historical imagery.

\item[\code{ViewerOptions}] character vector containing any of 'sunlight', 'historicalimagery' and 'streetview'. By specifing any of these you enable the special viewing modes. 

\item[\code{longitude}] numeric. Define the longitude, for type of 'lookat' this is the longitude of the focal point, for type of 'camera' this is actual longitude of the camera position.

\item[\code{latitude}] numeric. Define the latitude, for type of 'lookat' this is the latitude of the focal point, for type of 'camera' this is actual latitude of the camera position.
\item[\code{altitude}] numeric. Define the altitude, for type of 'lookat' this is the altitude of the focal point, for type of 'camera' this is actual altitude of the camera position.
\item[\code{altitudeMode}] character. One of "clampToGround", "relativeToGround", "absolute", "clampToSeaFloor", "relativeToSeaFloor", defining how the altitude of the view is intrepreted 
\item[\code{heading}] numeric, 0 - 360. Define the direction that the view will be facing.

\item[\code{tilt}] numeric, 0-180) for type of 'lookat' define the up/down angle to focal point, for type of 'camera' define the up/down angle of the camera. 0 is looking straight down, 90 looks at horizon, 180 is looking straight up into space.

\item[\code{range}] numeric. Used only with type of 'lookat', Define how far from the focal point you would like the camera to belarge the label will appear. (Default: 1)
\item[\code{roll}] numeric, +180 to -180. Used only with type of 'camera', Define the camera left/right roll angle. 

\end{ldescription}
\end{Arguments}
%
\begin{Note}\relax
\strong{viewid}\\{}
Do not attempt to define more than one AbstractView for the same viewid.

\end{Note}
%
\begin{Author}\relax
Brent Cameron\\{}
Department of Fisheries and Oceans Canada\\{}
Population Ecology Division
\end{Author}
%
\begin{References}\relax
\Rhref{https://developers.google.com/kml/documentation/kmlreference\#abstractview}{KML abstractView Reference}\\{}
\Rhref{https://developers.google.com/kml/documentation/kmlreference\#camera}{KML camera Reference}\\{}
\Rhref{https://developers.google.com/kml/documentation/kmlreference\#lookat}{KML lookat Reference}

\end{References}
%
\begin{Examples}
\begin{ExampleCode}
#Create the kml object
twoViewskml = RKmlObject()
#add an Astract view of type lookat. look east, at 70 degrees 
#upward, 1000 meters away from the defined position(latitude, 
#longitude, altitude) 
twoViewskml$addAbstractView(type = "lookat", viewid = "sable_lookat", latitude = 43.94, longitude = -60.1, altitude = 1000, altitudeMode = "relativeToGround", heading = 90, tilt = 70, range = 1000)

#add an Astract view of type camera. position camera at defined position(latitude, 
#longitude, altitude) then point camera east look up 50 degrees and roll to the left slightly  
twoViewskml$addAbstractView(type = "camera", viewid = "sable_camera", latitude = 43.94, longitude = -60.1, altitude = 5000, altitudeMode = "relativeToGround", heading = 90,  tilt = 50, roll = 40)
#Add 2 folders each with one of the above abstractviews associated with it. Click on folder in tree to fly to view.
twoViewskml$addFolder("lookat", name = "lookat example", description = "Click on this folder to fly to the lookat example", AbstractView = "sable_lookat", open = 1)
twoViewskml$addFolder("camera", name = "camera example", description = "Click on this folder to fly to the camera example", AbstractView = "sable_camera", open = 1)
#twoViewskml$preview()
\end{ExampleCode}
\end{Examples}
\inputencoding{utf8}
\HeaderA{addBalloonStyle}{Add BalloonStyle to KML Object}{addBalloonStyle}
%
\begin{Description}\relax
After creating a kml object, this method defines a style that can be used by Balloon features within the object. When a feature uses a balloon, be sure that the feature's styleUrl matches the desired styleid argument. See references for documentation on how to effectively use the text argument.
\end{Description}
%
\begin{Usage}
\begin{verbatim}
yourKMLobj$addBalloonStyle(styleid, bgColor, textColor, text, displayMode)
\end{verbatim}
\end{Usage}
%
\begin{Arguments}
\begin{ldescription}
\item[\code{styleid}] 
Mandatory. Define the id for this style. kml features will need to reference this id to use the defined style. You may choose an id that has already been defined in other styles types such as LineStyle or PolyStyle.

\item[\code{bgColor}] 
Define the background color of the balloon. This can be one of the predefined colors(), color as hex, or color as rgb object. (Default: "white")

\item[\code{textColor}] 
Define the text color of the balloon. This can be one of the predefined colors(), color as hex, or color as rgb object. (Default: "black")

\item[\code{text}] 
Define what text will appear in the balloon. This can be customized with each feature by using markup within the text. See references for more details. (Default: A CDATA markup that resembles the default kml balloon)

\item[\code{displayMode}] 
Either "display" or "hide". Choose weather balloon will be displayed. (Default: "display")

\end{ldescription}
\end{Arguments}
%
\begin{Value}
Does not return a value, simply adds style to kml objects style list
\end{Value}
%
\begin{Note}\relax
\strong{styleid}\\{}
Do not attempt to define more than one balloonstyle for the same styleid.
\\{}
\strong{text}\\{}
For advanced text arguments, wrap html in CDATA tags. You can use packages to build html such as R2HTML(uses headers that are not required by CDATA) or, if you are comfortable with html, just add your html text inside CDATA enclosures.\\{}
\code{text = ''<![CDATA[}\\{}
\code{<b><font color='\#CC0000' size='+3'>\$[name]</font></b>}\\{}
\code{<br/><br/>}\\{}
\code{<font face='Courier'>\$[description]</font>}\\{}
\code{]]>''}


\end{Note}
%
\begin{Author}\relax
Brent Cameron\\{}
Department of Fisheries and Oceans Canada\\{}
Population Ecology Division
\end{Author}
%
\begin{References}\relax
\Rhref{https://developers.google.com/kml/documentation/kmlreference\#balloonstyle}{KML balloonStyle Reference}
\end{References}
%
\begin{Examples}
\begin{ExampleCode}

mykml = RKmlObject()
customtext = "This is the $[name]. <br/> Click on this to look at $[description]. <br/> See references for more custom text insert  options."
 
mykml$addBalloonStyle(styleid = "colorstyle1", bgColor = "violetred4", textColor = rgb(.2, .2, 1))
mykml$addBalloonStyle(styleid = "textstyle1", bgColor = "black", textColor = "white", text = customtext, displayMode = "display")
 
x = data.frame(cbind(c(45.9178, 46.807), c(-59.967, -60.321), c("colorstyle1", "textstyle1"), c("Color Example", "Text Example"), c("Louisbourg", "Neil's Harbour" )))
names(x) = c("lat", "lon", "styleUrl", "name", "description")
  
mykml$addPoint(x)
#mykml$preview()

\end{ExampleCode}
\end{Examples}
\inputencoding{utf8}
\HeaderA{addFolder, getFolder, removeFolder}{Add Return and Remove Folder to and from KML file Structure}{addFolder, getFolder, removeFolder}
%
\begin{Description}\relax
Manipulate the directory structure for your kml. Folders can be created within other folders. Folders must have unique fid within the same directory. 
\end{Description}
%
\begin{Usage}
\begin{verbatim}
yourKMLobj$addFolder(fid, name, silent, visibility, open, atomauthor, atomlinkhref, address, xalAddressDetails, phoneNumber, Snippet, description, AbstractView, TimeStamp, TimeSpanStart, TimeSpanEnd, styleUrl, Region, ExtendedData)
yourKMLobj$getFolder(fid)
yourKMLobj$removeFolder(fid)

\end{verbatim}
\end{Usage}
%
\begin{Arguments}
\begin{ldescription}

\item[\code{fid}] 
Mandatory. Define the id for this folder. The id will be used to specify where features are added in the file structure.


\item[\code{name}] Character string or NULL. This is the text that will be displayed in the file tree. (Default = NULL) 
\item[\code{silent}] TRUE if you do not wish to see messages, False if you would like folder opperations to be displayed. (Default = TRUE) 
\item[\code{visibility}] Boolean (0-invisible or 1-visible). Specify if the features contained by this folder are visible. (Default = 1)
\item[\code{open}]  Boolean (0-closed or 1-open). Specify if the folder is open or closed in the kml object tree. (Default = 0)
\item[\code{atomauthor}]  Character string or NULL. Text identifying an author relevant to the topic. See Ascription Elements under References (Default = NULL) 
\item[\code{atomlinkhref}] Character string or NULL. Text identifying a web link relevant to the topic. See Ascription Elements under References (Default = NULL) 
\item[\code{address}] Character string or NULL. Specify an unstructured address that is associated with the folder. (Default = NULL) 
\item[\code{xalAddressDetails}] Character string or NULL. Specify a structured address in eXtensible Address Language(See xal:AddressDetails under References) that is associated with the folder. (Default = NULL) 
\item[\code{phoneNumber}]  Character string or NULL. Text identifying a phone number, useful for mobile apps. (Default = NULL) 
\item[\code{Snippet}] Character string or NULL. Supply details in addition to description. (Default = NULL)
\item[\code{description}] Character string or NULL. Supply details, may contain CDATA. See NOTES section for more info. (Default = NULL)
\item[\code{AbstractView}] Character string or NULL. The id of the desired AbstractView. Must create AbstractView with addAbstractView() function. (Default = NULL)
\item[\code{TimeStamp}]  Character string of date-time in one of the following formats: (YYYY)(YYYY-MM)(YYYY-MM-DD)(YYYY-MM-DDThh:mm:ssZ)(YYYY-MM-DDThh:mm:ss). Used to create kml timeseries. (Default = NULL)
\item[\code{TimeSpanStart}]  Character string of date-time in one of the following formats(overrides TimeStamp): (YYYY)(YYYY-MM)(YYYY-MM-DD)(YYYY-MM-DDThh:mm:ssZ)(YYYY-MM-DDThh:mm:ss). Used to create timeseries. (Default = NULL)
\item[\code{TimeSpanEnd}]  Character string of date-time in one of the following formats(overrides TimeStamp): (YYYY)(YYYY-MM)(YYYY-MM-DD)(YYYY-MM-DDThh:mm:ssZ)(YYYY-MM-DDThh:mm:ss). Used to create timeseries. (Default = NULL) 
\item[\code{styleUrl}] Character string or NULL. The id of the desired Style. Must create a style with one of the addStyle() or interactiveStyle() functions. (Default = NULL)
\item[\code{Region}] Currently not supported
\item[\code{ExtendedData}] Currently not supported

\end{ldescription}
\end{Arguments}
%
\begin{Note}\relax

\strong{description}\\{}
For advanced description arguments, wrap html in CDATA tags. You can use packages to build html such as R2HTML(uses headers that are not required by CDATA) or, if you are comfortable with html, just add your html text inside CDATA enclosures. \\{}
\code{text = ''<!\bsl{}[CDATA\bsl{}[ }\\{}
\code{<b><font color="\#CC0000" size='+3'>Example Text</font></b>}\\{}
\code{<br/><br/>}\\{}
\code{<font face='Courier'>Example Text</font>}\\{}
\code{]]>''}\\{}
\\{}
See also: \Rhref{https://developers.google.com/kml/documentation/kmlreference\#descriptionexample}{Description Example}

\end{Note}
%
\begin{Author}\relax
Brent Cameron \\{}
Department of Fisheries and Oceans Canada \\{}
Population Ecology Division 
\end{Author}
%
\begin{References}\relax
\Rhref{https://developers.google.com/kml/documentation/kmlreference\#folder}{KML Folder Reference } \\{}
\Rhref{https://developers.google.com/kml/documentation/kmlreference\#sampleattribution}{Ascription Elements}\\{} 
\Rhref{http://www.schemacentral.com/sc/kml22/e-xal_AddressDetails.html}{xal:AddressDetails Details}

\end{References}
%
\begin{Examples}
\begin{ExampleCode}

mykml = RKmlObject()

mykml$addFolder(fid = "2013", name = "2013") #Demonstrate adding folder
mykml$getFolder("2013")$addFolder(fid = "survey_data", name = "survey data") #Demonstrate adding nested folder

##Demonstrate variable assignment of folder object##
sur2013 = mykml$getFolder("2013")$getFolder("survey_data") 
sur2013$addFolder(fid = "harbour")

mykml$getFolder("2013")$removeFolder("survey_data") #Demonstrate remove folder
mykml$getFolder("2013")$addFolder(fid = "survey_data", name = "survey data")
sur2013 = mykml$getFolder("2013")$getFolder("survey_data")
sur2013$addFolder(fid = "ports", name = "ports")

x = data.frame(cbind(c(45.9178, 46.807), c(-59.967, -60.321)))
names(x) = c("lat", "lon")

sur2013$getFolder("ports")$addPoint(x) #Demonstrate add data to folder

#mykml$preview()

\end{ExampleCode}
\end{Examples}
\inputencoding{utf8}
\HeaderA{addGroundOverlay}{Add a Ground Overlay to kml}{addGroundOverlay}
%
\begin{Description}\relax
Add groundoverlay to a kml object. The function uses rgdal package to read positional metadata from the supplied image. If the image has no positional metadata attached then you will need to supply the east and west longitude bounds plus the north and south latitude bounds.
\end{Description}
%
\begin{Usage}
\begin{verbatim}
addGroundOverlay(fn = pathtogeoimage)
addGroundOverlay(fn = pathtosimpleimage, east = maxlon, west = minlon, north = maxlat, south = minlat)
\end{verbatim}
\end{Usage}
%
\begin{Arguments}
\begin{ldescription}

\item[\code{fn}] Mandatory. Specify the path to the image.
\item[\code{x}] Optional dataframe. Allow batch adding of groundoverlays, column names the same as variable names. 
\item[\code{east}]  Define the eastern most edge of the image. Only required if fn has no positional metadata. 
\item[\code{west}]  Define the western most edge of the image. Only required if fn has no positional metadata. 
\item[\code{north}]  Define the northern most edge of the image. Only required if fn has no positional metadata. 
\item[\code{west}]  Define the southern most edge of the image. Only required if fn has no positional metadata. 


\item[\code{drawOrder}]  numeric, set the order of the image rendering. (Default:NULL) 
\item[\code{color}] Specify a color to blend with the image. This can be one of the predefined colors(), color as hex, or color as rgb object. (Default: NULL)
\item[\code{transparency}] numeric. Set the transparency for the color. 0.0 - 1.0, fully opaque:0, solid:1. (Default: 1)
\item[\code{altitude}]  numeric, meters above/below altiudeMode. 'clamp' altitudeModes ignore altitude. (Default: 0)
\item[\code{altitudeMode}] one of "clampToGround", "relativeToGround", "absolute", "clampToSeaFloor", "relativeToSeaFloor". (Default:"clampToGround")
\item[\code{name}] Character string or NULL. What the feature will be called in the graphics environment. (Default:NULL) 
\item[\code{visibility}] Boolean (0-invisible or 1-visible). Specify if these features are visible. (Default: 1)
\item[\code{open}]  Boolean (0-closed or 1-open). Specify if the feature is open or closed in the kml object tree. (Default: 0)
\item[\code{atomauthor}]  Character string or NULL. Text identifying an author relevant to the topic. See Ascription Elements under References (Default: NULL) 
\item[\code{atomlinkhref}] Character string or NULL. Text identifying a web link relevant to the topic. See Ascription Elements under References (Default: NULL) 
\item[\code{address}] Character string or NULL. Specify an unstructured address that is associated with the overlay. (Default:NULL) 
\item[\code{xalAddressDetails}] Character string or NULL. Specify an structured address in eXtensible Address Language(sSee xal:AddressDetails under References) that is associated with the overlay. (Default:NULL) 
\item[\code{phoneNumber}]  Character string or NULL. Text identifying a phone number, useful for mobile apps. (Default:NULL) 
\item[\code{Snippet}] Character string or NULL. Supply details in addition to description. (Default:NULL)
\item[\code{description}] Character string or NULL. Supply details, may contain CDATA. See NOTES section for more info. (Default:NULL)
\item[\code{AbstractView}] Character string or NULL. The id of the desired AbstractView. Must create AbstractView with addAbstractView() function. (Default:NULL)
\item[\code{TimeStamp}]  Character string of date-time in one of the following formats: (YYYY)(YYYY-MM)(YYYY-MM-DD)(YYYY-MM-DDThh:mm:ssZ)(YYYY-MM-DDThh:mm:ss). Used to create kml timeseries. (Default:NULL)
\item[\code{TimeSpanStart}]  Character string of date-time in one of the following formats(overides TimeStamp): (YYYY)(YYYY-MM)(YYYY-MM-DD)(YYYY-MM-DDThh:mm:ssZ)(YYYY-MM-DDThh:mm:ss). Used to create timeseries. (Default:NULL)
\item[\code{TimeSpanEnd}]  Character string of date-time in one of the following formats(overrides TimeStamp): (YYYY)(YYYY-MM)(YYYY-MM-DD)(YYYY-MM-DDThh:mm:ssZ)(YYYY-MM-DDThh:mm:ss). Used to create timeseries. (Default:NULL) 
\item[\code{styleUrl}] Character string or NULL. The id of the desired Style. Must create a style with one of the addStyle() or interactiveStyle() functions. (Default:NULL)
\item[\code{Region}] Currently not supported
\item[\code{ExtendedData}] Currently not supported
\item[\code{inFolder}] Allows the adding of data to a specified folder, (EXPERIMENTAL) Usage: inFolder = 'this\bsl{}that'   

\end{ldescription}
\end{Arguments}
%
\begin{Note}\relax

\strong{description}\\{}
For advanced description arguments, wrap html in CDATA tags. You can use packages to build html such as R2HTML(uses headers that are not required by CDATA) or, if you are comfortable with html, just add your html text inside CDATA enclosures. \\{}
\code{text = ''<!\bsl{}[CDATA\bsl{}[ }\\{}
\code{<b><font color="\#CC0000" size='+3'>Example Text</font></b>}\\{}
\code{<br/><br/>}\\{}
\code{<font face='Courier'>Example Text</font>}\\{}
\code{]]>''}\\{}
\\{}
See also: \Rhref{https://developers.google.com/kml/documentation/kmlreference\#descriptionexample}{Description Example}

\end{Note}
%
\begin{Author}\relax
Brent Cameron \\{}
Department of Fisheries and Oceans Canada \\{}
Population Ecology Division 
\end{Author}
%
\begin{References}\relax
\Rhref{https://developers.google.com/kml/documentation/kmlreference\#groundoverlay}{KML groundoverlay Reference} \\{}
\Rhref{https://developers.google.com/kml/documentation/kmlreference\#sampleattribution}{Ascription Elements}\\{} 
\Rhref{http://www.schemacentral.com/sc/kml22/e-xal_AddressDetails.html}{xal:AddressDetails}

\end{References}
%
\begin{Examples}
\begin{ExampleCode}

mykml = RKmlObject()
mykml$addAbstractView(type = "lookat", viewid = "Rview", latitude = 45.5, longitude = -62, range = 1000000)

fn = file.path(R.home(), "doc", "html", "logo.jpg")
mykml$addGroundOverlay(fn = fn, east = -63, west = -61, north = 46.5, south = 44.5, 
  transparency = .3, color = "red", name = "RedR", AbstractView = "Rview" )

#mykml$preview()

\end{ExampleCode}
\end{Examples}
\inputencoding{utf8}
\HeaderA{addIconStyle}{Add IconStyle to KML Object}{addIconStyle}
%
\begin{Description}\relax
After creating a kml object, this method defines a style that can be used by Icon features within the object. When a feature uses a icon, be sure that the feature's styleUrl matches the desired styleid argument. 
\end{Description}
%
\begin{Usage}
\begin{verbatim}

yourKMLobj$addIconStyle(styleid, href, color, scale, heading, xunits, yunits, x, y, colorMode)
\end{verbatim}
\end{Usage}
%
\begin{Arguments}
\begin{ldescription}

\item[\code{styleid}] 
Mandatory. Define the id for this style. kml features will need to reference this id to use the defined style. You may choose an id that has already been defined in other styles types such as LineStyle or PolyStyle.


\item[\code{href}] character. Path to image, can be HTTP address or a local file.(Default: Google's default pushpin)


\item[\code{color}] Define the color to be blended with the image. This can be one of the predefined colors(), color as hex, or color as rgb object. (Default: NULL)
\item[\code{transparency}] numeric. Set the transparency for the color. 0.0 - 1.0, fully opaque:0, solid:1. (Default: 1)
\item[\code{scale}] numeric. Define how large the image will appear. (Default: 1)
\item[\code{heading}] numeric. Define a rotation for the image between 0 and 360. (Default: 0)
\item[\code{xunits}] character. Define units for specifying the x anchor point in the image. One of 'fraction', 'pixels' or 'insetPixels'. (Default: "fraction")
\item[\code{x}] numeric. For fraction units a value from 0 to 1. For pixels and insetPixels units a value from 1 to x resolution. (Default: 0.5)
\item[\code{yunits}] character. Define units for specifying the y anchor point in the image. One of 'fraction', 'pixels' or 'insetPixels'. (Default: "fraction")
\item[\code{y}] numeric. For fraction units a value from 0 to 1. For pixels and insetPixels units a value from 1 to y resolution. (Default: 0.5)
\item[\code{colorMode}] character. One of 'normal' or 'random'. Random will sudo-randomly apply a color based on the color argument, use color = white for true random. (Default: "normal")
\end{ldescription}
\end{Arguments}
%
\begin{Note}\relax
\strong{colorMode}\\{}
see \Rhref{https://developers.google.com/kml/documentation/kmlreference\#colormode}{KML colorMode Reference}
\\{}

\strong{styleid}\\{}
Do not attempt to define more than one iconstyle for the same styleid.

\end{Note}
%
\begin{Author}\relax
Brent Cameron\\{}
Department of Fisheries and Oceans Canada\\{}
Population Ecology Division
\end{Author}
%
\begin{References}\relax
\Rhref{https://developers.google.com/kml/documentation/kmlreference\#iconstyle}{KML iconStyle Reference}
\end{References}
%
\begin{Examples}
\begin{ExampleCode}

mykml = RKmlObject()

fishpath = "http://maps.google.com/mapfiles/ms/micons/fishing.png" 
mykml$addIconStyle(styleid = "iconstyle1", href = fishpath, color = "red", scale = 5, heading = 180)
mykml$addIconStyle(styleid = "iconstyle2", href = fishpath, color = "blue", scale = 1, heading = 0)
 
x = data.frame(cbind(c(45.9178, 46.807), c(-59.967, -60.321), c("iconstyle1", "iconstyle2"), c("red fishing", "blue fishing"), c("Louisbourg", "Neil's Harbour" )))
names(x) = c("lat", "lon", "styleUrl", "name", "description")
  
mykml$addPoint(x)
#mykml$preview()

\end{ExampleCode}
\end{Examples}
\inputencoding{utf8}
\HeaderA{addLabelStyle}{Add LabelStyle to KML Object}{addLabelStyle}
%
\begin{Description}\relax
After creating a kml object, this method defines a style that can be used by Label features within the object. When a feature uses a label, be sure that the feature's styleUrl matches the desired styleid argument. See references for documentation on how to effectively use the text argument.
\end{Description}
%
\begin{Usage}
\begin{verbatim}

yourKMLobj$addLabelStyle(styleid, color, transparency, colorMode, scale)
\end{verbatim}
\end{Usage}
%
\begin{Arguments}
\begin{ldescription}

\item[\code{styleid}] 
Mandatory. Define the id for this style. kml features will need to reference this id to use the defined style. You may choose an id that has already been defined in other styles types such as LineStyle or PolyStyle.



\item[\code{color}] Define the color of the label. This can be one of the predefined colors(), color as hex, or color as rgb object. (Default: "red")
\item[\code{transparency}] numeric. Set the transparency for the color. 0.0 - 1.0, fully opaque:0, solid:1. (Default: 1)
\item[\code{colorMode}] character. One of 'normal' or 'random'. Random will sudo-randomly apply a color based on the color argument, use color = white for true random. (Default: "normal")
\item[\code{scale}] numeric. Define how large the label will appear. (Default: 1)


\end{ldescription}
\end{Arguments}
%
\begin{Note}\relax
\strong{colorMode}\\{}
see \Rhref{https://developers.google.com/kml/documentation/kmlreference\#colormode}{KML colormode Reference}
\\{}

\strong{styleid}\\{}
Do not attempt to define more than one iconstyle for the same styleid.

\end{Note}
%
\begin{Author}\relax
Brent Cameron\\{}
Department of Fisheries and Oceans Canada\\{}
Population Ecology Division
\end{Author}
%
\begin{References}\relax
\Rhref{https://developers.google.com/kml/documentation/kmlreference\#labelstyle}{KML labelStyle Reference}
\end{References}
%
\begin{Examples}
\begin{ExampleCode}

mykml = RKmlObject()


mykml$addLabelStyle(styleid = "labelstyle1", color = "red", transparency = .2, scale = 3)
mykml$addLabelStyle(styleid = "labelstyle2", color = "green", transparency = 1, scale = 1)
 
x = data.frame(cbind(c(45.9178, 46.807), c(-59.967, -60.321), c("labelstyle1", "labelstyle2"), c("Example1", "Example2"), c("Louisbourg", "Neil's Harbour" )))
names(x) = c("lat", "lon", "styleUrl", "name", "description")
  
mykml$addPoint(x)
#mykml$preview()

\end{ExampleCode}
\end{Examples}
\inputencoding{utf8}
\HeaderA{addLineString}{Add LineString features to a kml}{addLineString}
%
\begin{Description}\relax
After creating a kml object, you can add LineString features with this function so they are viewable in the graphics window of the google API. A dataframe x must be suplied to this function. At a minimum, x must contain columns pid, lat and lon. Each unique pid represents an individual LineString. To style these features, use addLineStyle and match styleid to styleUrl. 
\end{Description}
%
\begin{Usage}
\begin{verbatim}
yourKMLobj$addLineString(x) 
yourKMLobj$addLineString(x, altitude, styleUrl)
\end{verbatim}
\end{Usage}
%
\begin{Arguments}
\begin{ldescription}

\item[\code{x}]  Mandatory. A dataframe with columns pid, lat and lon. You may also want to specify altitude for each lat/lon. There must be 2 or more records for each unique pid. x may also contain additional columns. All of the listed optional arguments can be supplied in the function call to apply to records of x, or can be a column of x(column name = argument name) with unique values at each unique pid to apply to that LineString only.  

\item[\code{altitude}] numeric with 1 or more elements cycled by pid. Meters above/below altiudeMode. 'clamp' altitudeModes ignore altitude. (Default: 0)
\item[\code{extrude}] boolean (0-not extrude or 1-extrude) with 1 or more elements cycled by pid. Choose weather to extrude(connect) the LineString to the ground. If you wish to extrude, altitudeMode must be one of relativeToGround, relativeToSeaFloor, or absolute. (Default: 1)
\item[\code{tessellate}] boolean (0-not tesselate or 1-tesselate) with 1 or more elements cycled by pid. Choose weather LineString will tesselate(follow terrain). If you wish to tesselate, altitudeMode must be one of clampToGround or clampToSeaFloor. (Default: 1)
\item[\code{drawOrder}]  numeric with 1 or more elements cycled by pid, set the order of the feature rendering. (Default:NULL) 
\item[\code{altitudeMode}] character with 1 or more elements cycled by pid, one of "clampToGround", "relativeToGround", "absolute", "clampToSeaFloor", "relativeToSeaFloor". (Default:"clampToGround")
\item[\code{name}] character with 1 or more elements cycled by pid, what the feature will be called in the graphics environment. (Default:NULL) 
\item[\code{visibility}] boolean (0-invisible or 1-visible) with 1 or more elements cycled by pid. Specify if this feature will be visible. (Default: 1)
\item[\code{open}]  boolean (0-closed or 1-open) with 1 or more elements cycled by pid. Specify if the feature is open or closed in the kml object tree. (Default: 0)
\item[\code{atomauthor}]  character with 1 or more elements cycled by pid, text identifying an author relevant to the topic. See Ascription Elements under References (Default: NULL) 
\item[\code{atomlinkhref}] character with 1 or more elements cycled by pid, text identifying a web link relevant to the topic. See Ascription Elements under References (Default: NULL) 
\item[\code{address}] character with 1 or more elements cycled by pid, specify an unstructured address that is associated with the lineString. (Default:NULL) 
\item[\code{xalAddressDetails}] character with 1 or more elements cycled by pid, specify a structured address in eXtensible Address Language(See xal:AddressDetails under References) that is associated with the line. (Default:NULL) 
\item[\code{phoneNumber}]  character with 1 or more elements cycled by pid, represent a phone number, useful for mobile apps. (Default:NULL) 
\item[\code{Snippet}] character with 1 or more elements cycled by pid, supply details in addition to description. (Default:NULL)
\item[\code{description}] character with 1 or more elements cycled by pid, supply details relevant to the feature. May contain CDATA. See NOTES section for more info. (Default:NULL)
\item[\code{AbstractView}] character with 1 or more elements cycled by pid, the id of the desired AbstractView. Must create AbstractView with addAbstractView() function. (Default:NULL)
\item[\code{TimeStamp}]  character with 1 or more elements cycled by pid, define a moment of date-time in one of the following formats: (YYYY)(YYYY-MM)(YYYY-MM-DD)(YYYY-MM-DDThh:mm:ssZ)(YYYY-MM-DDThh:mm:ss). Used to create kml timeseries. (Default:NULL)
\item[\code{TimeSpanStart}]  character with 1 or more elements cycled by pid, define the start of a span of date-time in one of the following formats(overides TimeStamp): (YYYY)(YYYY-MM)(YYYY-MM-DD)(YYYY-MM-DDThh:mm:ssZ)(YYYY-MM-DDThh:mm:ss). Used to create timeseries. (Default:NULL)
\item[\code{TimeSpanEnd}]  character with 1 or more elements cycled by pid, define the end of a span of date-time in one of the following formats(overrides TimeStamp): (YYYY)(YYYY-MM)(YYYY-MM-DD)(YYYY-MM-DDThh:mm:ssZ)(YYYY-MM-DDThh:mm:ss). Used to create timeseries. (Default:NULL) 
\item[\code{styleUrl}] character with 1 or more elements cycled by pid, the id of the desired Style. Must create a style with one of the addStyle() or interactiveStyle() functions. (Default:NULL)
\item[\code{Region}] Currently not supported
\item[\code{ExtendedData}] Currently not supported
\item[\code{inFolder}] Allows the adding of data to a specified folder, (EXPERIMENTAL) Usage: inFolder = 'this\bsl{}that'   
You can add styles directly by defining any of the following variables. This is however discouraged as a new style is created for each row of x. These will be ignored if styleid is defined.   
\item[\code{icon\_color, icon\_href, icon\_transparency, icon\_scale, icon\_heading, icon\_xunits, icon\_x, icon\_yunits, icon\_y, icon\_colorMode, bal\_bgColor, bal\_textColor, bal\_text,  bal\_displayMode, label\_color, label\_transparency, label\_colorMode, label\_scale, line\_color, line\_transparency, line\_width, line\_outerColor, line\_outerTransparency, line\_outerPortion, line\_colorMode, line\_labelVisibility	}] You can define these but it is POOR FORM!!! see icon, balloon, label, and line style documentation for values.   

\end{ldescription}
\end{Arguments}
%
\begin{Note}\relax

\strong{description}\\{}
For advanced description arguments, wrap html in CDATA tags. You can use packages to build html such as R2HTML(uses headers that are not required by CDATA) or, if you are comfortable with html, just add your html text inside CDATA enclosures. \\{}
\code{text = ''<!\bsl{}[CDATA\bsl{}[ }\\{}
\code{<b><font color="\#CC0000" size='+3'>Example Text</font></b>}\\{}
\code{<br/><br/>}\\{}
\code{<font face='Courier'>Example Text</font>}\\{}
\code{]]>''}\\{}
\\{}
See also: \Rhref{https://developers.google.com/kml/documentation/kmlreference\#descriptionexample}{Description Example}

\end{Note}
%
\begin{Author}\relax
Brent Cameron \\{}
Department of Fisheries and Oceans Canada \\{}
Population Ecology Division 
\end{Author}
%
\begin{References}\relax
\Rhref{https://developers.google.com/kml/documentation/kmlreference\#linestring}{KML lineString Reference} \\{}
\Rhref{https://developers.google.com/kml/documentation/kmlreference\#sampleattribution}{Ascription Elements}\\{} 
\Rhref{http://www.schemacentral.com/sc/kml22/e-xal_AddressDetails.html}{xal:AddressDetails Details}
\end{References}
%
\begin{Examples}
\begin{ExampleCode}

mykml = RKmlObject()

pid = c(1, 1, 1, 2, 2, 6, 6, 6, 6)
lat = c(44, 44.1, 44, 44.2, 44.2, 44, 44.3, 44.3, 44)
lon = c(-60, -59.9, -59.8, -60, -59.8, -60.4, -60, -59.8, -59.4)
altitude = c(1000, 1000, 500, 500, 2000, 2000, 1500, 500, 2000)
x = cbind(pid, lat, lon, altitude)

mykml$addLineString(x, altitudeMode = "relativeToGround")

#Create styles
mykml$addLineStyle(styleid = "linestyle1", color = "red", transparency = .5, width = 100, outerColor = "orange", outerTransparency = 1, outerPortion = .5)
mykml$addLineStyle(styleid = "linestyle2", color = "yellow", transparency = 1, width = 50, outerColor = "green", outerTransparency = 1, outerPortion = .2)
mykml$addLineStyle(styleid = "linestyle3", color = "#0000ff", transparency = 1)

#Shift lats up to see style applied next to non style
lat = lat + .02

x = cbind(pid, lat,  lon)
altitude = c(1000, 500, 0)
style = c("linestyle1", "linestyle2", "linestyle3") #Cycles with pid
mykml$addLineString(x, styleUrl = style, altitude = altitude, altitudeMode = "relativeToGround")
#OR apply 1 altitude to each pid
#mykml$addLineString(x, styleUrl = style, altitude = 1000, altitudeMode = "relativeToGround")

##All optional arguments can be supplied in the function call and will cycle by pid
#mykml$preview()
\end{ExampleCode}
\end{Examples}
\inputencoding{utf8}
\HeaderA{addLineStyle}{Add LineStyle to KML Object}{addLineStyle}
%
\begin{Description}\relax
After creating a kml object, this method defines a style that can be used by LineString features within the object. When a feature is a LineString, be sure that the feature's styleUrl matches the desired styleid argument. 
\end{Description}
%
\begin{Usage}
\begin{verbatim}
yourKMLobj$addLineStyle(styleid, color, transparency, width, outerColor, outerTransparency, outerPortion, colorMode, labelVisibility)
\end{verbatim}
\end{Usage}
%
\begin{Arguments}
\begin{ldescription}

\item[\code{styleid}] 
Mandatory. Define the id for this style. kml features will need to reference this id to use the defined style. You may choose an id that has already been defined in other styles types such as LabelStyle or PolyStyle.



\item[\code{color}] Define the color of the line. This can be one of the predefined colors(), color as hex, or color as rgb object. (Default: NULL)
\item[\code{transparency}] numeric. Set the transparency for the color. 0.0 - 1.0, fully opaque:0, solid:1. (Default: 1)
\item[\code{width}] numeric. Define the line width. If a single color line is used width is in pixels, if a deul band line is used width is in meters. (Default: 1)
\item[\code{outerColor}] Define the outer color band only if a deul band is desired. This can be one of the predefined colors(), color as hex, or color as rgb object. (Default: NULL)
\item[\code{outerTransparency}] numeric. Set the transparency for the outerColor. 0.0 - 1.0, fully opaque:0, solid:1. (Default: NULL)
\item[\code{outerPortion}] numeric. Define the portion of the width is the outer portion. 0.0 - 1.0. (Default: NULL)
\item[\code{colorMode}] character. One of 'normal' or 'random'. Random will sudo-randomly apply a color based on the color argument, use color = white for true random. (Default: "normal")
\item[\code{labelVisibility}] Boolean (0-invisible or 1-visible). Specify if a label should be displayed. Label value taken from name argument. (Default: 0)
\end{ldescription}
\end{Arguments}
%
\begin{Note}\relax
\strong{colorMode}\\{}
see \Rhref{https://developers.google.com/kml/documentation/kmlreference\#colormode}{color Mode}\\{}

\strong{styleid}\\{}
Do not attempt to define more than one linestyle for the same styleid.

\end{Note}
%
\begin{Author}\relax
Brent Cameron\\{}
Department of Fisheries and Oceans Canada\\{}
Population Ecology Division
\end{Author}
%
\begin{References}\relax
\Rhref{https://developers.google.com/kml/documentation/kmlreference\#linestyle}{KML lineStyle Reference}
\end{References}
%
\begin{Examples}
\begin{ExampleCode}

mykml = RKmlObject()

pid = c(1, 1, 1, 2, 2, 6, 6, 6, 6)
lat = c(44, 44.1, 44, 44.2, 44.2, 44, 44.3, 44.3, 44)
lon = c(-60, -59.9, -59.8, -60, -59.8, -60.4, -60, -59.8, -59.4)
x = cbind(pid, lat, lon)

mykml$addLineString(x)

#Create styles
mykml$addLineStyle(styleid = "linestyle1", color = "red", transparency = .5, width = 100, outerColor = "orange", outerTransparency = 1, outerPortion = .5)
mykml$addLineStyle(styleid = "linestyle2", color = "yellow", transparency = 1, width = 50, outerColor = "green", outerTransparency = 1, outerPortion = .2)
mykml$addLineStyle(styleid = "linestyle3", color = "#0000ff", transparency = 1, labelVisibility = 1)

#Shift lats up to see style applied next to non style
lat = lat + .02

styleUrl = c("linestyle1", "linestyle1", "linestyle1", "linestyle2", "linestyle2", "linestyle3", "linestyle3", "linestyle3", "linestyle3")
x = cbind(pid, lat, lon, styleUrl)
mykml$addLineString(x, name = "hello world")

#Same as above but without redundant styleURL
#x = cbind(pid, lat,  lon)
#style = c("linestyle1", "linestyle2", "linestyle3") #Cycles with pid
#mykml$addLineString(x, styleUrl = style)
#OR apply 1 altitude to each pid
#mykml$addLineString(x, styleUrl = style, altitude = 1000)

##All optional arguments can be supplied in the function call and will cycle by pid
#mykml$preview()

\end{ExampleCode}
\end{Examples}
\inputencoding{utf8}
\HeaderA{addListStyle}{Add ListStyle to KML Object}{addListStyle}
%
\begin{Description}\relax
After creating a kml object, this method defines a style that can be used by containers. When a container uses a ListStyle, be sure that the feature's styleUrl matches the desired styleid argument. See references for documentation on how to effectively use the text argument.
\end{Description}
%
\begin{Usage}
\begin{verbatim}

  
yourKMLobj$addListStyle(styleid, listItemTpye, bgColor)
\end{verbatim}
\end{Usage}
%
\begin{Arguments}
\begin{ldescription}

\item[\code{styleid}] 
Mandatory. Define the id for this style. kml containers will need to reference this id to use the defined style. You may choose an id that has already been defined in other styles types such as LineStyle or PolyStyle.


\item[\code{listItemType}] Define the way elements of a container using this style are displayed. This can be one of check, checkOffOnly, checkHideChildren or radioFolder. (Default: "check)
\item[\code{bgColor}] Define the background color. This can be one of the predefined colors(), color as hex, or color as rgb object. (Default: "white")

\end{ldescription}
\end{Arguments}
%
\begin{Note}\relax

\strong{styleid}\\{}
Do not attempt to define more than one liststle for the same styleid.

\end{Note}
%
\begin{Author}\relax
Brent Cameron\\{}
Department of Fisheries and Oceans Canada\\{}
Population Ecology Division
\end{Author}
%
\begin{References}\relax
\Rhref{https://developers.google.com/kml/documentation/kmlreference\#polystyle}{KML polyStyle Reference}
\end{References}
%
\begin{Examples}
\begin{ExampleCode}

mykml = RKmlObject()

#Create a radio style to allow only one element of a container to be displayed at a time
mykml$addListStyle(styleid = "test", bgColor = "green", listItemType = "radioFolder")
mykml$addPolyStyle(styleid = "mystyle2", color = "yellow", transparency = 1, fill = 0)
mykml$addFolder(fid = "new", styleUrl = "test")
mykml$getFolder("new")$addFolder("1")
mykml$getFolder("new")$addFolder("2")
mykml$getFolder("new")$addFolder("3")
 
#mykml$preview()

\end{ExampleCode}
\end{Examples}
\inputencoding{utf8}
\HeaderA{addNetworkLink}{Add NetworkLink}{addNetworkLink}
%
\begin{Description}\relax
After creating a kml object, you can add a link to an external kml or kmz file. 
\end{Description}
%
\begin{Usage}
\begin{verbatim}
yourKMLobj$addNetworkLink(href)
yourKMLobj$addNetworkLink(href, refreshVisibility, flyToView, refreshMode, refreshInterval, viewRefreshMode, viewRefreshTime, viewBoundScale, viewFormat, httpQuery, name, visibility, open, atomauthor, atomlinkhref, address, xalAddressDetails, phoneNumber, Snippet, description, AbstractView, TimeStamp, TimeSpanStart, TimeSpanEnd, styleUrl, Region, ExtendedData)
\end{verbatim}
\end{Usage}
%
\begin{Arguments}
\begin{ldescription}

\item[\code{href}] Mandatory. character, a local or network address that points to the kml or kmz file to link this kml.

\item[\code{refreshVisibility}] boolean (0-user controlled or 1-link controlled). Choose if you wish to control the visibility of features of a networklink or you will except the visibility of the link each time it is refreshed.
\item[\code{flToView}] boolean (0-user controlled or 1-link controlled). Choose if you wish to fly to the location of the lookAt elements of the networklink initially and on each refresh.
\item[\code{refreshMode}] character, one of "onChange, "onInterval"", or "onExpire".
\item[\code{refreshInterval}] numeric. How many seconds untill next refresh. Only use when refreshMode = "onInterval".
\item[\code{viewRefreshMode}] character, one of "never"", "onStop"", "onRequest" or "onRegion". Specifies how the link is refreshed when the "camera" changes.
\item[\code{viewRefreshTime}] numeric. How many seconds after stopping untill view is refreshed. Only use when viewRefreshMode = "onStop".
\item[\code{viewBoundScale}] numeric. How much of the current view will be fetched from the server onStop. <1- less than curent screen will be fetched, >1 more than current screen will be fetched. Scales the query (usually BBOX) 
\item[\code{viewFormat}] character. Specifies the format of the query string that is appended to the Link's <href> before the file is fetched. See Notes. 
\item[\code{httpQuery}] character. Additional fetch options added to the query. "[clientVersion]","[kmlVersion]","[clientName]" and "[language]" are supported. 

\item[\code{name}] character, what the feature will be called in the graphics environment. (Default:NULL) 
\item[\code{visibility}] boolean (0-invisible or 1-visible). Specify if this feature will be visible. (Default: 1)
\item[\code{open}]  boolean (0-closed or 1-open). Specify if the feature is open or closed in the kml object tree. (Default: 0)
\item[\code{atomauthor}]  character, text identifying an author relevant to the topic. See Ascription Elements under References (Default: NULL) 
\item[\code{atomlinkhref}] character, text identifying a web link relevant to the topic. See Ascription Elements under References (Default: NULL) 
\item[\code{address}] character, specify an unstructured address that is associated with the link. (Default:NULL) 
\item[\code{xalAddressDetails}] character, specify a structured address in eXtensible Address Language(See xal:AddressDetails under References) that is associated with the link. (Default:NULL) 
\item[\code{phoneNumber}]  character, represent a phone number, useful for mobile apps. (Default:NULL) 
\item[\code{Snippet}] character, supply details in addition to description. (Default:NULL)
\item[\code{description}] character, supply details relevant to the feature. May contain CDATA. See NOTES section for more info. (Default:NULL)
\item[\code{AbstractView}] character, the id of the desired AbstractView. Must create AbstractView with addAbstractView() function. (Default:NULL)
\item[\code{TimeStamp}]  character, define a moment of date-time in one of the following formats: (YYYY)(YYYY-MM)(YYYY-MM-DD)(YYYY-MM-DDThh:mm:ssZ)(YYYY-MM-DDThh:mm:ss). Used to create kml timeseries. (Default:NULL)
\item[\code{TimeSpanStart}]  character, define the start of a span of date-time in one of the following formats(overides TimeStamp): (YYYY)(YYYY-MM)(YYYY-MM-DD)(YYYY-MM-DDThh:mm:ssZ)(YYYY-MM-DDThh:mm:ss). Used to create timeseries. (Default:NULL)
\item[\code{TimeSpanEnd}]  character, define the end of a span of date-time in one of the following formats(overrides TimeStamp): (YYYY)(YYYY-MM)(YYYY-MM-DD)(YYYY-MM-DDThh:mm:ssZ)(YYYY-MM-DDThh:mm:ss). Used to create timeseries. (Default:NULL) 
\item[\code{styleUrl}] character, the id of the desired Style. Must create a style with one of the addStyle() or interactiveStyle() functions. (Default:NULL)
\item[\code{Region}] Currently not supported
\item[\code{ExtendedData}] Currently not supported
\item[\code{inFolder}] Allows the adding of data to a specified folder, (EXPERIMENTAL) Usage: inFolder = 'this\bsl{}that'   


\end{ldescription}
\end{Arguments}
%
\begin{Note}\relax


\strong{description}\\{}
For advanced description arguments, wrap html in CDATA tags. You can use packages to build html such as R2HTML(uses headers that are not required by CDATA) or, if you are comfortable with html, just add your html text inside CDATA enclosures. \\{}
\code{text = ''<!\bsl{}[CDATA\bsl{}[ }\\{}
\code{<b><font color="\#CC0000" size='+3'>Example Text</font></b>}\\{}
\code{<br/><br/>}\\{}
\code{<font face='Courier'>Example Text</font>}\\{}
\code{]]>''}\\{}
\\{}
See also: \Rhref{https://developers.google.com/kml/documentation/kmlreference\#descriptionexample}{Description Example}

\strong{viewFormat}\\{}
\code{if viewRefreshMode of onStop, defaults to BBOX=[bboxWest],[bboxSouth],[bboxEast],[bboxNorth]}\\{}

You can also specify a custom set of viewing parameters. See \Rhref{https://developers.google.com/kml/documentation/kmlreference\#Link}{KML Link Reference}




\end{Note}
%
\begin{Author}\relax
Brent Cameron \\{}
Department of Fisheries and Oceans Canada \\{}
Population Ecology Division 
\end{Author}
%
\begin{References}\relax
\Rhref{https://developers.google.com/kml/documentation/kmlreference\#networklink}{KML networklink Reference}\\{}
\Rhref{https://developers.google.com/kml/documentation/kmlreference\#Link}{KML Link Reference}\\{}
\Rhref{http://geochalkboard.wordpress.com/2007/09/20/the-kml-behind-network-links/}{Network Link Additional Info}\\{}
\Rhref{https://developers.google.com/kml/documentation/kmlreference\#sampleattribution}{Ascription Elements}\\{} 
\Rhref{http://www.schemacentral.com/sc/kml22/e-xal_AddressDetails.html}{xal:AddressDetails}

\end{References}
%
\begin{Examples}
\begin{ExampleCode}
### Simple example ###
mykml = RKmlObject()
mykml$addFolder("MyNetworkLinks", name = "NetworkLinks")

#Environment Canada GoeMet service
path2kmz = "www.ec.gc.ca/meteo-weather/C0D9B3D8-D256-407D-A68F-C606D703105E/GeoMet-E.kmz" 

mykml$getFolder("MyNetworkLinks")$addNetworkLink(href = path2kmz, name= "GeoMet")
#mykml$preview()

### End Simple Example ###





\end{ExampleCode}
\end{Examples}
\inputencoding{utf8}
\HeaderA{addNetworkLinkControl}{Add NetworkLink Control}{addNetworkLinkControl}
%
\begin{Description}\relax
Use this function to add control over how this kml will be fetched by other kml/kmz files. This is usually only needed if you plan on hosting this kml. Calling this function overwrites any previously created NetworkLink Controls
\end{Description}
%
\begin{Usage}
\begin{verbatim}
yourKMLobj$addNetworkLinkControl() #EMPTY networkLinkControl
yourKMLobj$addNetworkLinkControl(minRefreshPeriod, maxSessionLength, cookie, message, linkName, linkDescription, linkSnippet, expires, update, AbstractView)
\end{verbatim}
\end{Usage}
%
\begin{Arguments}
\begin{ldescription}

\item[\code{minRefreshPeriod}] numeric, seconds that need to pass before this kml can be fetched again. Helps throtle network.


\item[\code{maxSessionLength}] numeric, seconds to keep the kml linked. A value of -1 means remain open indefinitly. (Default: -1)
\item[\code{cookie}] character. text appended to the URL query on the next refresh of the network link. This is usefull whenlinking to cgi scripts that can use conditional file delivery.
\item[\code{message}] character. If specified, a pop-up will apear when a client links to this kml with the contents of the text.
\item[\code{linkName}] character. Define the name of the networkLink which will appear in the client's file tree.
\item[\code{linkDescription}] character. Define a description for this link that will appear in the client's file tree.
\item[\code{linkSnippet}] character. Define more information about this link in addition to linkDescription.
\item[\code{expires}] character. date/time (formats: (YYYY)(YYYY-MM)(YYYY-MM-DD)(YYYY-MM-DDThh:mm:ssZ)(YYYY-MM-DDThh:mm:ss) at which the link should be refreshed. This specification is used only when the the client kml has Link argument refreshMode = 'onExpire'. 
\item[\code{update}] character. xml text specifing a previously loaded kml and actions change, create or delete. The is for advanced users only. See KML networkLinkControll under References for a good description
\item[\code{AbstractView}] character. Reference to the id of an Abstract view that has already been created or will soon be created. This is where the clients camera will go when this kml is linked.
\end{ldescription}
\end{Arguments}
%
\begin{Author}\relax
Brent Cameron\\{}
Department of Fisheries and Oceans Canada\\{}
Population Ecology Division
\end{Author}
%
\begin{References}\relax
\Rhref{https://developers.google.com/kml/documentation/kmlreference\#networklinkcontrol}{KML networkLinkControll Reference}
\Rhref{http://geochalkboard.wordpress.com/2007/09/20/the-kml-behind-network-links/}{Network Link Aditional Info}
\end{References}
%
\begin{Examples}
\begin{ExampleCode}

## Changes view after 20 seconds

#Create the hosted kml
hostkml = RKmlObject()
#add an Astract view
hostkml$addAbstractView(type = "lookat", viewid = "sable_view", latitude = 43.9, longitude = -59.9, range = 100000)
hostkml$addNetworkLinkControl(minRefreshPeriod = 20, maxSessionLength = 60, message = "Welcome to the example Network Control kml. This kml will only stay linked for 60 seconds", linkName = "Sable Island", linkDescription = "You will soon fly to St.Paul's Island", linkSnippet = "buckle your seatbelt", AbstractView = "sable_view")
#hostkml$writekml("hostkml.kml")

#Create a client kml
clientkml = RKmlObject()

#kml client will attempt to refresh every second but will fail due to hosts minRefreshPeriod
clientkml$addNetworkLink(href = "hostkml.kml",flyToView = 1, refreshMode = "onInterval", refreshInterval = 1)
#clientkml$preview()

#Change the abstract view, should update in the preview of the clientkml after 20 seconds from the clientkml #linking to the host kml

hostkml$addAbstractView(type = "lookat", viewid = "stPaulsView", latitude = 47.2, longitude = -60.15, range = 10000)
hostkml$addNetworkLinkControl(minRefreshPeriod = 20, maxSessionLength = 60, message = "Welcome to the example Network Control kml. This kml will only stay linked for 60 seconds", linkName = "St.Paul's Island", linkDescription = "The link will expire in 40 seconds", linkSnippet = "watch your step", AbstractView = "stPaulsView")

#hostkml$writekml("hostkml.kml")


### End Example ###

\end{ExampleCode}
\end{Examples}
\inputencoding{utf8}
\HeaderA{addPoint}{Add Point Features}{addPoint}
%
\begin{Description}\relax
After creating a kml object or folder, you can add Point features with this function so they are viewable in the graphics window of the google API. A dataframe x must be suplied to this function. At a minimum, x must contain columns lat and lon. To style these features, use addIconStyle and/or addLabelSytle and match styleid to styleUrl. 
\end{Description}
%
\begin{Usage}
\begin{verbatim}
yourKMLobj$addPoint(x, ...)
yourKMLobj$addPoint(x, altitude, styleUrl, extrude, altitudeMode, name, visibility, open, atomauthor, atomlinkhref, address, xalAddressDetails, phoneNumber, Snippet, description, AbstractView, TimeStamp, TimeSpanStart, TimeSpanEnd, styleUrl, Region, ExtendedData)
\end{verbatim}
\end{Usage}
%
\begin{Arguments}
\begin{ldescription}

\item[\code{x}] Mandatory. A dataframe with columns lat and lon. x may also contain additional columns. All of the listed optional arguments can be supplied in the function call to apply to records of x, or can be a column of x (column name = argument name). This gives the ability to provide specific arguments to specific points.  
\item[\code{altitude}] numeric. Meters above/below altiudeMode. 'clamp' altitudeModes ignore altitude. (Default: 0)
\item[\code{extrude}] boolean (0-not extrude or 1-extrude). Choose weather to extrude(connect) the Point to the ground. If you wish to extrude, altitudeMode must be one of relativeToGround, relativeToSeaFloor, or absolute. (Default: 1)
\item[\code{altitudeMode}] character, one of "clampToGround", "relativeToGround", "absolute", "clampToSeaFloor", "relativeToSeaFloor". (Default:"clampToGround")
\item[\code{name}] character, what the feature will be called in the graphics environment. (Default:NULL) 
\item[\code{visibility}] boolean (0-invisible or 1-visible). Specify if this feature will be visible. (Default: 1)
\item[\code{open}]  boolean (0-closed or 1-open). Specify if the feature is open or closed in the kml object tree. (Default: 0)
\item[\code{atomauthor}]  character, text identifying an author relevant to the topic. See Ascription Elements under References(Default: NULL) 
\item[\code{atomlinkhref}] character, text identifying a web link relevant to the topic. See Ascription Elements under References(Default: NULL) 
\item[\code{address}] character, specify an unstructured address that is associated with the point. (Default:NULL) 
\item[\code{xalAddressDetails}] character, specify a structured address in eXtensible Address Language(See xal:AddressDetails under References) that is associated with the point. (Default:NULL) 
\item[\code{phoneNumber}]  character, represent a phone number, useful for mobile apps. (Default:NULL) 
\item[\code{Snippet}] character, supply details in addition to description. (Default:NULL)
\item[\code{description}] character, supply details relevant to the feature. May contain CDATA. See NOTES section for more info. (Default:NULL)
\item[\code{AbstractView}] character, the id of the desired AbstractView. Must create AbstractView with addAbstractView() function. (Default:NULL)
\item[\code{TimeStamp}]  character, define a moment of date-time in one of the following formats: (YYYY)(YYYY-MM)(YYYY-MM-DD)(YYYY-MM-DDThh:mm:ssZ)(YYYY-MM-DDThh:mm:ss). Used to create kml timeseries. (Default:NULL)
\item[\code{TimeSpanStart}]  character, define the start of a span of date-time in one of the following formats(overides TimeStamp): (YYYY)(YYYY-MM)(YYYY-MM-DD)(YYYY-MM-DDThh:mm:ssZ)(YYYY-MM-DDThh:mm:ss). Used to create timeseries. (Default:NULL)
\item[\code{TimeSpanEnd}]  character, define the end of a span of date-time in one of the following formats(overrides TimeStamp): (YYYY)(YYYY-MM)(YYYY-MM-DD)(YYYY-MM-DDThh:mm:ssZ)(YYYY-MM-DDThh:mm:ss). Used to create timeseries. (Default:NULL) 
\item[\code{styleUrl}] character, the id of the desired Style. Must create a style with one of the addStyle() or interactiveStyle() functions. (Default:NULL)
\item[\code{Region}] Currently not supported
\item[\code{ExtendedData}] Currently not supported
\item[\code{inFolder}] Allows the adding of data to a specified folder, (EXPERIMENTAL) Usage: inFolder = 'this\bsl{}that'   
You can add styles directly by defining any of the following variables. This is however discouraged as a new style is created for each row of x. These will be ignored if styleid is defined.   
\item[\code{icon\_color, icon\_href, icon\_transparency, icon\_scale, icon\_heading, icon\_xunits, icon\_x, icon\_yunits, icon\_y, icon\_colorMode, bal\_bgColor, bal\_textColor, bal\_text,	bal\_displayMode, label\_color, label\_transparency, label\_colorMode, label\_scale, line\_color, line\_transparency, line\_width, line\_outerColor, line\_outerTransparency, line\_outerPortion, line\_colorMode, line\_labelVisibility	}] You can define these but it is POOR FORM!!! see icon, balloon, label, and line style documentation for values.   
\end{ldescription}
\end{Arguments}
%
\begin{Note}\relax

\strong{description}\\{}
For advanced description arguments, wrap html in CDATA tags. You can use packages to build html such as R2HTML(uses headers that are not required by CDATA) or, if you are comfortable with html, just add your html text inside CDATA enclosures. \\{}
\code{text = ''<!\bsl{}[CDATA\bsl{}[ }\\{}
\code{<b><font color="\#CC0000" size='+3'>Example Text</font></b>}\\{}
\code{<br/><br/>}\\{}
\code{<font face='Courier'>Example Text</font>}\\{}
\code{]]>''}\\{}
\\{}
See also: \Rhref{https://developers.google.com/kml/documentation/kmlreference\#descriptionexample}{Description Example}

\end{Note}
%
\begin{Author}\relax
Brent Cameron \\{}
Department of Fisheries and Oceans Canada \\{}
Population Ecology Division 
\end{Author}
%
\begin{References}\relax
\Rhref{https://developers.google.com/kml/documentation/kmlreference\#point}{KML Point Reference} \\{}
\Rhref{https://developers.google.com/kml/documentation/kmlreference\#descriptionexample}{Description Example}\\{}
\Rhref{https://developers.google.com/kml/documentation/kmlreference\#sampleattribution}{Ascription Elements}\\{} 
\Rhref{http://www.schemacentral.com/sc/kml22/e-xal_AddressDetails.html}{xal:AddressDetails}
\end{References}
%
\begin{Examples}
\begin{ExampleCode}

mykml = RKmlObject()


imagepath = "http://maps.google.com/mapfiles/kml/paddle/wht-circle-lv.png" 
mykml$addIconStyle(styleid = "mystyle", color = "salmon2", href = imagepath, scale = .5, colorMode = "random")
mykml$addLabelStyle(styleid = "mystyle", color = "#000000", transparency = .8, scale = 1)

lat = c(44, 44.1, 44, 44.2, 44.2, 44, 44.3, 44.3, 44)
lon = c(-60, -59.9, -59.8, -60, -59.8, -60.4, -60, -59.8, -59.4)
extrude = c(0, 0, 1, 1, 1, 0, 0, 1, 0)
altitude = c(1000, 800, 1000, 1000, 1200, 5000, 5000, 10000, 10000)
name = c("p1","p2","p3","p4","p5","p6","p7","p8","p9")
x = cbind(lat, lon, extrude, altitude, name)
  
mykml$addPoint(x, styleUrl = "mystyle", description = "This is an example description applied to all points", altitudeMode = "relativeToGround")
#mykml$preview()
\end{ExampleCode}
\end{Examples}
\inputencoding{utf8}
\HeaderA{addPolygon}{Add Polygon features to a kml}{addPolygon}
%
\begin{Description}\relax
After creating a kml object, you can add Polygon features with this function so they are viewable in the graphics window of the google API. A data-frame x must be supplied to this function to define the the polygons and geography. A second data-frame y may also be supplied to define an inner region of a polygon for cutting out. At a minimum, x and y must contain columns pid, lat and lon.  Each unique pid represents an individual Polygon. To style these features, use addLineStyle and match styleid to styleUrl. 
\end{Description}
%
\begin{Usage}
\begin{verbatim}
yourKMLobj$addPolygon(x)
yourKMLobj$addLineString(x, y, altitude, styleUrl)
\end{verbatim}
\end{Usage}
%
\begin{Arguments}
\begin{ldescription}

\item[\code{x}] Mandatory. A data-frame with columns pid, lat and lon. There must be 3 or more records for each unique pid. x may also contain additional columns. All of the listed optional arguments can be supplied in the function call to apply to records of x, or can be a column of x(column name = argument name) with unique values at each unique pid to apply to that LineString only.  
\item[\code{y}] A data-frame with columns pid, lat and lon that define an inner region to remove from polygons in x, matched on pid. There must be 3 or more records for each unique pid.
\item[\code{altitude}] numeric with 1 or more elements cycled by pid. Meters above/below altiudeMode. 'clamp' altitudeModes ignore altitude. (Default: 0)
\item[\code{extrude}] boolean (0-not extrude or 1-extrude) with 1 or more elements cycled by pid. Choose weather to extrude(connect) the LineString to the ground. If you wish to extrude, altitudeMode must be one of relativeToGround, relativeToSeaFloor, or absolute. (Default: 1)
\item[\code{tessellate}] boolean (0-not tessellate or 1-tessellate) with 1 or more elements cycled by pid. Choose weather Polygon will tessellate(follow terrain). If you wish to tessellate, altitudeMode must be one of clampToGround or clampToSeaFloor. (Default: 1)
\item[\code{altitudeMode}] character with 1 or more elements cycled by pid, one of "clampToGround", "relativeToGround", "absolute", "clampToSeaFloor", "relativeToSeaFloor". (Default:"clampToGround")
\item[\code{name}] character with 1 or more elements cycled by pid, what the feature will be called in the graphics environment. (Default:NULL) 
\item[\code{visibility}] boolean (0-invisible or 1-visible) with 1 or more elements cycled by pid. Specify if this feature will be visible. (Default: 1)
\item[\code{open}]  boolean (0-closed or 1-open) with 1 or more elements cycled by pid. Specify if the feature is open or closed in the kml object tree. (Default: 0)
\item[\code{atomauthor}]  character with 1 or more elements cycled by pid, text identifying an author relevant to the topic. See Ascription Elements under References(Default: NULL) 
\item[\code{atomlinkhref}] character with 1 or more elements cycled by pid, text identifying a web link relevant to the topic. See Ascription Elements under References (Default: NULL) 
\item[\code{address}] character with 1 or more elements cycled by pid, specify an unstructured address that is associated with the polygon. (Default:NULL) 
\item[\code{xalAddressDetails}] character with 1 or more elements cycled by pid, specify a structured address in eXtensible Address Language(See xal:AddressDetails under References) that is associated with the polygon. (Default:NULL) 
\item[\code{phoneNumber}]  character with 1 or more elements cycled by pid, represent a phone number, useful for mobile apps. (Default:NULL) 
\item[\code{Snippet}] character with 1 or more elements cycled by pid, supply details in addition to description. (Default:NULL)
\item[\code{description}] character with 1 or more elements cycled by pid, supply details relevant to the feature. May contain CDATA. See NOTES section for more info. (Default:NULL)
\item[\code{AbstractView}] character with 1 or more elements cycled by pid, the id of the desired AbstractView. Must create AbstractView with addAbstractView() function. (Default:NULL)
\item[\code{TimeStamp}]  character with 1 or more elements cycled by pid, define a moment of date-time in one of the following formats: (YYYY)(YYYY-MM)(YYYY-MM-DD)(YYYY-MM-DDThh:mm:ssZ)(YYYY-MM-DDThh:mm:ss). Used to create kml time-series. (Default:NULL)
\item[\code{TimeSpanStart}]  character with 1 or more elements cycled by pid, define the start of a span of date-time in one of the following formats(overrides TimeStamp): (YYYY)(YYYY-MM)(YYYY-MM-DD)(YYYY-MM-DDThh:mm:ssZ)(YYYY-MM-DDThh:mm:ss). Used to create time-series. (Default:NULL)
\item[\code{TimeSpanEnd}]  character with 1 or more elements cycled by pid, define the end of a span of date-time in one of the following formats(overrides TimeStamp): (YYYY)(YYYY-MM)(YYYY-MM-DD)(YYYY-MM-DDThh:mm:ssZ)(YYYY-MM-DDThh:mm:ss). Used to create time-series. (Default:NULL) 
\item[\code{styleUrl}] character with 1 or more elements cycled by pid, the id of the desired Style. Must create a style with one of the addStyle() or interactiveStyle() functions. (Default:NULL)
\item[\code{Region}] Currently not supported
\item[\code{ExtendedData}] Currently not supported
\item[\code{inFolder}] Allows the adding of data to a specified folder, (EXPERIMENTAL) Usage: inFolder = 'this\bsl{}that'      

You can add styles directly by defining any of the following variables. This is however discouraged as a new style is created for each row of x. These will be ignored if styleid is defined.   
\item[\code{icon\_color, icon\_href, icon\_transparency, icon\_scale, icon\_heading, icon\_xunits, icon\_x, icon\_yunits, icon\_y, icon\_colorMode, bal\_bgColor, bal\_textColor, bal\_text,  bal\_displayMode, label\_color, label\_transparency, label\_colorMode, label\_scale, line\_color, line\_transparency, line\_width, line\_outerColor, line\_outerTransparency, line\_outerPortion, line\_colorMode, line\_labelVisibility	}] You can define these but it is POOR FORM!!! see icon, balloon, label, and line style documentation for values.   

\end{ldescription}
\end{Arguments}
%
\begin{Note}\relax


\strong{description}\\{}
For advanced description arguments, wrap html in CDATA tags. You can use packages to build html such as R2HTML(uses headers that are not required by CDATA) or, if you are comfortable with html, just add your html text inside CDATA enclosures. \\{}
\code{text = ''<!\bsl{}[CDATA\bsl{}[ }\\{}
\code{<b><font color="\#CC0000" size='+3'>Example Text</font></b>}\\{}
\code{<br/><br/>}\\{}
\code{<font face='Courier'>Example Text</font>}\\{}
\code{]]>''}\\{}
\\{}
See also: \Rhref{https://developers.google.com/kml/documentation/kmlreference\#descriptionexample}{Description Example}


\end{Note}
%
\begin{Author}\relax
Brent Cameron \\{}
Department of Fisheries and Oceans Canada \\{}
Population Ecology Division 
\end{Author}
%
\begin{References}\relax
\Rhref{https://developers.google.com/kml/documentation/kmlreference\#polygon}{KML polygon Reference} \\{}
\Rhref{https://developers.google.com/kml/documentation/kmlreference\#sampleattribution}{Ascription Elements}\\{} 
\Rhref{http://www.schemacentral.com/sc/kml22/e-xal_AddressDetails.html}{xal:AddressDetails}
\end{References}
%
\begin{Examples}
\begin{ExampleCode}

mykml = RKmlObject()

pid = c(1, 1, 1, 2, 2, 2, 6, 6, 6, 6)
lat = c(44, 44.1, 44, 44.2, 44.2, 44.1, 44, 44.1, 44.1, 44)
lon = c(-60, -59.9, -59.8, -60.2, -60.1, -60, -60.5, -60.5, -60.2, -60.2)
altitude=c(2000,2000,4000,4000,5000,2000,0,0,0,0)
x = cbind(pid, lat, lon, altitude)

mykml$addPolygon(x)

  
#Create styles
mykml$addPolyStyle(styleid = "polystyle1", color = "red", transparency = .5)
mykml$addPolyStyle(styleid = "polystyle2", color = "yellow", transparency = 1, fill = 0)
mykml$addPolyStyle(styleid = "polystyle3", color = "yellow", transparency = 1, outline = 0)

#Shift lats up to see style applied next to non style
lat = lat + .2

x = cbind(pid, lat, lon)

style = c("polystyle1", "polystyle2", "polystyle3") #Cycles with pid
##Add altitude
altitude = c(2000, 4000, 8000) #Cycles with pid
mykml$addPolygon(x, styleUrl = style, altitude = altitude)

##All optional arguments can be supplied in the function call and will cycle by pid
#mykml$preview()
\end{ExampleCode}
\end{Examples}
\inputencoding{utf8}
\HeaderA{addPolyStyle}{Add PolyStyle to KML Object}{addPolyStyle}
%
\begin{Description}\relax
After creating a kml object, this method defines a style that can be used by Polygon features within the object. When a feature uses a polygon, be sure that the feature's styleUrl matches the desired styleid argument. See references for documentation on how to effectively use the text argument.
\end{Description}
%
\begin{Usage}
\begin{verbatim}

  
yourKMLobj$addLabelStyle(styleid, color, transparency, colorMode, fill, outline)
\end{verbatim}
\end{Usage}
%
\begin{Arguments}
\begin{ldescription}

\item[\code{styleid}] 
Mandatory. Define the id for this style. kml features will need to reference this id to use the defined style. You may choose an id that has already been defined in other styles types such as LineStyle or PolyStyle.



\item[\code{color}] Define the color of the Polygon. This can be one of the predefined colors(), color as hex, or color as rgb object. (Default: "red")
\item[\code{transparency}] numeric. Set the transparency for the color. 0.0 - 1.0, fully opaque:0, solid:1. (Default: 1)
\item[\code{colorMode}] character. One of 'normal' or 'random'. Random will sudo-randomly apply a color based on the color argument, use color = white for true random. (Default: "normal")
\item[\code{fill}] boolean (0-no fill or 1-fill). Choose weather to fill the polygon with the color. (Default: 1)
\item[\code{outline}] boolean (0-no outline or 1-outline). Choose weather to draw an outline arround the polygon. (Default: 1)

\end{ldescription}
\end{Arguments}
%
\begin{Note}\relax
\strong{colorMode}\\{}
see \Rhref{https://developers.google.com/kml/documentation/kmlreference\#colormode}{KML colorMode Reference}
\\{}

\strong{styleid}\\{}
Do not attempt to define more than one polystyle for the same styleid.

\end{Note}
%
\begin{Author}\relax
Brent Cameron\\{}
Department of Fisheries and Oceans Canada\\{}
Population Ecology Division
\end{Author}
%
\begin{References}\relax
\Rhref{https://developers.google.com/kml/documentation/kmlreference\#polystyle}{KML polyStyle Reference}
\end{References}
%
\begin{Examples}
\begin{ExampleCode}

mykml = RKmlObject()

#Create styles
mykml$addPolyStyle(styleid = "mystyle1", color = "red", transparency = .5)
mykml$addPolyStyle(styleid = "mystyle2", color = "yellow", transparency = 1, fill = 0)
mykml$addPolyStyle(styleid = "mystyle3", color = "yellow", transparency = 1, outline = 0)

pid = c(1, 1, 1, 2, 2, 2, 6, 6, 6, 6)
lat = c(44, 44.1, 44, 44.2, 44.2, 44.1, 44, 44.1, 44.1, 44)
lon = c(-60, -59.9, -59.8, -60.2, -60.1, -60, -60.5, -60.5, -60.2, -60.2)

x = cbind(pid, lat, lon)

style = c("mystyle1", "mystyle2", "mystyle3") #Cycles with pid
altitude = c(2000, 4000, 8000) #Cycles with pid


mykml$addPolygon(x, styleUrl = style, altitude = altitude)

mykml$addLabelStyle(styleid = "mystyle1", color = "red", transparency = .2, scale = 3)
mykml$addLabelStyle(styleid = "mystyle2", color = "green", transparency = 1, scale = 1)
 
#mykml$preview()

\end{ExampleCode}
\end{Examples}
\inputencoding{utf8}
\HeaderA{addScreenOverlay}{Add a Screen Overlay to kml}{addScreenOverlay}
%
\begin{Description}\relax
Add a screen overlay to a kml object. This will be usefull when adding titles or legends to your kml. 
\end{Description}
%
\begin{Usage}
\begin{verbatim}
addScreenOverlay(fn = pathtogeoimage)
addGroundOverlay(fn = pathtoimage, overlay_x, overlay_xunit, overlay_y, overlay_yunit, screen_x, screen_Y)
\end{verbatim}
\end{Usage}
%
\begin{Arguments}
\begin{ldescription}
\item[\code{fn}] Mandatory. Specify the path to the image.
\item[\code{overlay\_x}] numeric. Set the x axis anchor position on the image. 0.0 - 1.0 for 'fraction' units, 1 - max(x resolution) for 'pixel' or 'insetpixel' units. (Default: .5)
\item[\code{overlay\_y}] numeric. Set the x axis anchor position on the image. 0.0 - 1.0 for 'fraction' units, 1 - max(x resolution) for 'pixel' or 'insetpixel' units. (Default: .5)
\item[\code{overlay\_xunit}] character. One of 'fraction', 'pixels' or 'insetpixels'. Set the units for the overlay\_x argument. (Default: 'fraction')
\item[\code{overlay\_yunit}] character. One of 'fraction', 'pixels' or 'insetpixels'. Set the units for the overlay\_y argument. (Default: 'fraction')
\item[\code{screen\_x}] numeric. Set the x axis anchor position on the screen. 0.0 - 1.0 for 'fraction' units, 1 - max(x resolution) for 'pixel' or 'insetpixel' units. (Default: .5)
\item[\code{screen\_y}] numeric. Set the x axis anchor position on the screen. 0.0 - 1.0 for 'fraction' units, 1 - max(x resolution) for 'pixel' or 'insetpixel' units. (Default: .5)
\item[\code{screen\_xunit}] character. One of 'fraction', 'pixels' or 'insetpixels'. Set the units for the screen\_x argument. (Default: 'fraction')
\item[\code{screen\_yunit}] character. One of 'fraction', 'pixels' or 'insetpixels'. Set the units for the screen\_y argument. (Default: 'fraction')
\item[\code{rotation\_x}] numeric. Set the x axis screen position about which the image will be rotated. 0.0 - 1.0 for 'fraction' units, 1 - max(x resolution) for 'pixel' or 'insetpixel' units. (Default: .5)
\item[\code{rotation\_y}] numeric. Set the y axis screen position about which the image will be rotated. 0.0 - 1.0 for 'fraction' units, 1 - max(x resolution) for 'pixel' or 'insetpixel' units. (Default: .5)
\item[\code{rotation\_xunit}] character. One of 'fraction', 'pixels' or 'insetpixels'. Set the units for the rotation\_x argument. (Default: 'fraction')
\item[\code{rotation\_yunit}] character. One of 'fraction', 'pixels' or 'insetpixels'. Set the units for the rotation\_y argument. (Default: 'fraction')
\item[\code{size\_x}] numeric. Set the size of the image along the x axis. 0.0 - 1.0 for 'fraction' units, 1 - n for 'pixel' units. (Default: 1)
\item[\code{size\_y}] numeric. Set the size of the image along the y axis. 0.0 - 1.0 for 'fraction' units, 1 - n for 'pixel' units. (Default: 1)
\item[\code{size\_xunit}] character. One of 'fraction' or 'pixels'. Set the units for the size\_x argument. (Default: 'fraction')
\item[\code{size\_yunit}] character. One of 'fraction' or 'pixels'. Set the units for the size\_y argument. (Default: 'fraction') 
\item[\code{drawOrder}]  numeric, set the order of the image rendering. (Default:NULL) 
\item[\code{color}] Specify a color to blend with the image. This can be one of the predefined colors(), color as hex, or color as rgb object. (Default: NULL)
\item[\code{transparency}] numeric. Set the transparency for the color. 0.0 - 1.0, fully opaque:0, solid:1. (Default: 1)
\item[\code{rotation}] numeric. Set the transparency for the color. 0.0 - 1.0, fully opaque:0, solid:1. (Default: 1)
\item[\code{altitude}]  numeric, meters above/below altiudeMode. 'clamp' altitudeModes ignore altitude. (Default: 0)
\item[\code{altitudeMode}] one of "clampToGround", "relativeToGround", "absolute", "clampToSeaFloor", "relativeToSeaFloor". (Default:"clampToGround")
\item[\code{name}] Character string or NULL. What the feature will be called in the graphics environment. (Default:NULL) 
\item[\code{visibility}] Boolean (0-invisible or 1-visible). Specify if the feature will be visible. (Default: 1)
\item[\code{open}]  Boolean (0-closed or 1-open). Specify if the feature is open or closed in the kml object tree. (Default: 0)
\item[\code{atomauthor}]  Character string or NULL. Text identifying an author relevant to the topic. See Ascription Elements under References (Default: NULL) 
\item[\code{atomlinkhref}] Character string or NULL. Text identifying a web link relevant to the topic.  See Ascription Elements under References (Default: NULL) 
\item[\code{address}] Character string or NULL. Specify an unstructured address that is associated with the overlay. (Default:NULL) 
\item[\code{xalAddressDetails}] Character string or NULL. Specify an structured address in eXtensible Address Language(See xal:AddressDetails under References) that is associated with the overlay. (Default:NULL) 
\item[\code{phoneNumber}]  Character string or NULL. Text identifying a phone number, useful for mobile apps. (Default:NULL) 
\item[\code{Snippet}] Character string or NULL. Supply details in addition to description. (Default:NULL)
\item[\code{description}] Character string or NULL. Supply details, may contain CDATA. See NOTES section for more info. (Default:NULL)
\item[\code{AbstractView}] Character string or NULL. The id of the desired AbstractView. Must create AbstractView with addAbstractView() function. (Default:NULL)
\item[\code{TimeStamp}]  Character string of date-time in one of the following formats: (YYYY)(YYYY-MM)(YYYY-MM-DD)(YYYY-MM-DDThh:mm:ssZ)(YYYY-MM-DDThh:mm:ss). Used to create kml timeseries. (Default:NULL)
\item[\code{TimeSpanStart}]  Character string of date-time in one of the following formats(overides TimeStamp): (YYYY)(YYYY-MM)(YYYY-MM-DD)(YYYY-MM-DDThh:mm:ssZ)(YYYY-MM-DDThh:mm:ss). Used to create timeseries. (Default:NULL)
\item[\code{TimeSpanEnd}]  Character string of date-time in one of the following formats(overrides TimeStamp): (YYYY)(YYYY-MM)(YYYY-MM-DD)(YYYY-MM-DDThh:mm:ssZ)(YYYY-MM-DDThh:mm:ss). Used to create timeseries. (Default:NULL) 
\item[\code{styleUrl}] Character string or NULL. The id of the desired Style. Must create a style with one of the addStyle() or interactiveStyle() functions. (Default:NULL)
\item[\code{Region}] Currently not supported
\item[\code{ExtendedData}] Currently not supported
\item[\code{inFolder}] Allows the adding of data to a specified folder, (EXPERIMENTAL) Usage: inFolder = 'this\bsl{}that'   
\end{ldescription}
\end{Arguments}
%
\begin{Note}\relax

\strong{description}\\{}
For advanced description arguments, wrap html in CDATA tags. You can use packages to build html such as R2HTML(uses headers that are not required by CDATA) or, if you are comfortable with html, just add your html text inside CDATA enclosures. \\{}
\code{text = ''<!\bsl{}[CDATA\bsl{}[ }\\{}
\code{<b><font color="\#CC0000" size='+3'>Example Text</font></b>}\\{}
\code{<br/><br/>}\\{}
\code{<font face='Courier'>Example Text</font>}\\{}
\code{]]>''}\\{}
\\{}
See also: \Rhref{https://developers.google.com/kml/documentation/kmlreference\#descriptionexample}{Description Example}


\end{Note}
%
\begin{Author}\relax
Brent Cameron \\{}
Department of Fisheries and Oceans Canada \\{}
Population Ecology Division 
\end{Author}
%
\begin{References}\relax
\Rhref{https://developers.google.com/kml/documentation/kmlreference\#screenoverlay}{KML screenoverlay Reference} \\{}
\Rhref{https://developers.google.com/kml/documentation/kmlreference\#sampleattribution}{Ascription Elements}\\{} 
\Rhref{http://www.schemacentral.com/sc/kml22/e-xal_AddressDetails.html}{xal:AddressDetails}
\end{References}
%
\begin{Examples}
\begin{ExampleCode}
png(filename="figure.png", height=200, width=300, 
    bg="transparent")
plot(1, type="n", axes=FALSE, xlab="", ylab="")
legend(1, 1, legend = c("Hello", "World"), col=1:2, lwd=2, cex=3, xjust=0.5, yjust=0.5)
dev.off()
mykml = RKmlObject()
fn = file.path(getwd(), "figure.png")
mykml$addScreenOverlay(fn, size_x = .2, size_y = .2)

#mykml$preview()

\end{ExampleCode}
\end{Examples}
\inputencoding{utf8}
\HeaderA{addStyleMap}{Create Mouseover Effects with Style}{addStyleMap}
%
\begin{Description}\relax
With this function you can map mouseover and non-mouseover styles to a single style id. The style id can then be referenced by features that you wish to have a mouseover effect. 
\end{Description}
%
\begin{Usage}
\begin{verbatim}
addStyleMap(id, idn, idh)
\end{verbatim}
\end{Usage}
%
\begin{Arguments}
\begin{ldescription}
\item[\code{id}] 
Mandatory. Define style id. kml features will need to reference this id to use the defined style. You must choose a unique style id.
  
\item[\code{idn}] Mandatory. Reference the normal(non-highlighted) style id. If you have not yet created this style, you must created it in order for it to be linked. 
\item[\code{idh}] Mandatory. Reference the highlighted style id. If you have not yet created this style, you must created it in order for it to be linked. 
\end{ldescription}
\end{Arguments}
%
\begin{Author}\relax
Brent Cameron\\{}
Department of Fisheries and Oceans Canada\\{}
Population Ecology Division
\end{Author}
%
\begin{References}\relax
\Rhref{https://developers.google.com/kml/documentation/kmlreference\#stylemap}{KML styleMap Reference}
\end{References}
%
\begin{Examples}
\begin{ExampleCode}

mykml = RKmlObject()

#add styles. labelstyle1 will be the normal style, labelstyle2 will be the mouseover style
mykml$addLabelStyle(styleid = "labelstyle1", color = "red", transparency = .5, scale = 1)
mykml$addLabelStyle(styleid = "labelstyle2", color = "green", transparency = 1, scale = 2)

#make icon transparent(invisible) for the mouseover style
mykml$addIconStyle(styleid = "labelstyle2", color = "white", transparency = 0)

#add map style so mouseover effects can be acheived
mykml$addStyleMap(id = "my_mouseover_style", idn = "labelstyle1", idh = "labelstyle2")

#create points so styles can be visualized
x = data.frame(cbind(c(45.9178, 46.807), c(-59.967, -60.321), c("Example1", "Example2"), c("Louisbourg", "Neil's Harbour" )))
names(x) = c("lat", "lon", "name", "description")
mykml$addPoint(x, styleUrl = "my_mouseover_style")

#sample your creation
#mykml$preview()

\end{ExampleCode}
\end{Examples}
\inputencoding{utf8}
\HeaderA{interactiveBalloonStyle}{Interactively Create BalloonStyle}{interactiveBalloonStyle}
%
\begin{Description}\relax
This function is usually automatically entered by calling yourkmlobj\$styleBuilder(). However it can be called on it own, but you must supply an id.
\end{Description}
%
\begin{Usage}
\begin{verbatim}
interactiveBalloonStyle(id)
\end{verbatim}
\end{Usage}
%
\begin{Arguments}
\begin{ldescription}
\item[\code{id}] Mandatory. Define the id for this style. This id can be the same as other non-balloon style ids previously created.
\end{ldescription}
\end{Arguments}
%
\begin{Note}\relax
\strong{id}\\{}
Do not attempt to define more than one balloonstyle for the same id.

\end{Note}
%
\begin{Author}\relax
Brent Cameron\\{}
Department of Fisheries and Oceans Canada\\{}
Population Ecology Division
\end{Author}
%
\begin{Examples}
\begin{ExampleCode}

mykml = RKmlObject()

##ANSWER THE QUESTIONS. For this example the answers are supplied below. Typically the questions
##will be answered on the fly. 

## Not run: 
mykml$interactiveBalloonStyle(id = 'mystyle')
hello
black
red
Y

## End(Not run)

x = data.frame(cbind(c(45.9178, 46.807), c(-59.967, -60.321), c("Example1", "Example2"), c("Louisbourg", "Neil's Harbour" )))
names(x) = c("lat", "lon", "name", "description")
  
mykml$addPoint(x, styleUrl = "mystyle")
#mykml$preview()

\end{ExampleCode}
\end{Examples}
\inputencoding{utf8}
\HeaderA{interactiveIconStyle}{Interactively Create IconStyle}{interactiveIconStyle}
%
\begin{Description}\relax
This function is usually automatically entered by calling yourkmlobj\$styleBuilder(). However it can be called on its own, but you must supply an id.
\end{Description}
%
\begin{Usage}
\begin{verbatim}
interactiveIconStyle(id)
\end{verbatim}
\end{Usage}
%
\begin{Arguments}
\begin{ldescription}
\item[\code{id}] 
Mandatory. Define the id for this style. This id can be the same as other non-icon style ids previously created.

\end{ldescription}
\end{Arguments}
%
\begin{Note}\relax
\strong{id}\\{}
Do not attempt to define more than one IconStyle for the same id.

\end{Note}
%
\begin{Author}\relax
Brent Cameron\\{}
Department of Fisheries and Oceans Canada\\{}
Population Ecology Division
\end{Author}
%
\begin{Examples}
\begin{ExampleCode}

mykml = RKmlObject()

##ANSWER THE QUESTIONS. For this example the answers are supplied below. Typically the questions
##will be answered on the fly.
## Not run: 
mykml$interactiveIconStyle(id = "mystyle")
upload.wikimedia.org/wikipedia/commons/c/c1/Rlogo.png
2
180
Y
red
N
fraction
.5
fraction
.5

## End(Not run)
x = data.frame(cbind(c(45.9178, 46.807), c(-59.967, -60.321), c("Example1", "Example2"), c("Louisbourg", "Neil's Harbour" )))
names(x) = c("lat", "lon", "name", "description")
  
mykml$addPoint(x, styleUrl = "mystyle")
#mykml$preview()

\end{ExampleCode}
\end{Examples}
\inputencoding{utf8}
\HeaderA{interactiveLabelStyle}{Interactively Create LabelStyle}{interactiveLabelStyle}
%
\begin{Description}\relax
This function is usually automatically entered by calling yourkmlobj\$styleBuilder(). However it can be called on its own, but you must supply an id.
\end{Description}
%
\begin{Usage}
\begin{verbatim}
interactiveLabelStyle(id)
\end{verbatim}
\end{Usage}
%
\begin{Arguments}
\begin{ldescription}
\item[\code{id}] 
Mandatory. Define the id for this style. This id can be the same as other non-label style ids previously created.

\end{ldescription}
\end{Arguments}
%
\begin{Note}\relax
\strong{id}\\{}
Do not attempt to define more than one LabelStyle for the same id.

\end{Note}
%
\begin{Author}\relax
Brent Cameron\\{}
Department of Fisheries and Oceans Canada\\{}
Population Ecology Division
\end{Author}
%
\begin{Examples}
\begin{ExampleCode}

mykml = RKmlObject()

##ANSWER THE QUESTIONS. For this example the answers are supplied below. Typically the questions
##will be answered on the fly.
## Not run: 
mykml$interactiveLabelStyle(id = "mystyle")
orange
.3
3
N

## End(Not run)
x = data.frame(cbind(c(45.9178, 46.807), c(-59.967, -60.321), c("Example1", "Example2"), c("Louisbourg", "Neil's Harbour" )))
names(x) = c("lat", "lon", "name", "description")
  
mykml$addPoint(x, styleUrl = "mystyle")
#mykml$preview()

\end{ExampleCode}
\end{Examples}
\inputencoding{utf8}
\HeaderA{interactiveLineStyle}{Interactively Create LineStyle}{interactiveLineStyle}
%
\begin{Description}\relax
This function is usually automatically entered by calling yourkmlobj\$styleBuilder(). However it can be called on its own, but you must supply an id.
\end{Description}
%
\begin{Usage}
\begin{verbatim}
interactiveLineStyle(id)
\end{verbatim}
\end{Usage}
%
\begin{Arguments}
\begin{ldescription}
\item[\code{id}] 
Mandatory. Define the id for this style. This id can be the same as other non-line style ids previously created.

\end{ldescription}
\end{Arguments}
%
\begin{Note}\relax
\strong{id}\\{}
Do not attempt to define more than one LineStyle for the same id.

\end{Note}
%
\begin{Author}\relax
Brent Cameron\\{}
Department of Fisheries and Oceans Canada\\{}
Population Ecology Division
\end{Author}
%
\begin{Examples}
\begin{ExampleCode}

mykml = RKmlObject()

##ANSWER THE QUESTIONS. For this example the answers are supplied below. Typically the questions
##will be answered on the fly.
## Not run: 
mykml$interactiveLineStyle(id = "mystyle")
M
yellow
1
grey
.8
100
.8
N
Y

## End(Not run)
x = data.frame(cbind(c(1, 1, 1), c(45.9178, 46.3624, 46.807), c(-59.967,-60.114, -60.321)))
names(x) = c("pid", "lat", "lon")
  
mykml$addLineString(x, name= "fake road", description = "Louisbourg to Neil's Harbour", styleUrl = "mystyle", tessellate = 1 )
#mykml$preview()
#Notice the line disapears on close zoom, this is an error in Google Earth. This can be mitigated by adding more #points along the line.  
\end{ExampleCode}
\end{Examples}
\inputencoding{utf8}
\HeaderA{interactivePolyStyle}{Interactively Create PolyStyle}{interactivePolyStyle}
%
\begin{Description}\relax
This function is usually automatically entered by calling yourkmlobj\$styleBuilder(). However it can be called on its own, but you must supply an id.
\end{Description}
%
\begin{Usage}
\begin{verbatim}
interactivePolyStyle(id)
\end{verbatim}
\end{Usage}
%
\begin{Arguments}
\begin{ldescription}
\item[\code{id}] 
Mandatory. Define the id for this style. This id can be the same as other non-poly style ids previously created.

\end{ldescription}
\end{Arguments}
%
\begin{Note}\relax
\strong{id}\\{}
Do not attempt to define more than one LineStyle for the same id.

\end{Note}
%
\begin{Author}\relax
Brent Cameron\\{}
Department of Fisheries and Oceans Canada\\{}
Population Ecology Division
\end{Author}
%
\begin{Examples}
\begin{ExampleCode}

mykml = RKmlObject()

##ANSWER THE QUESTIONS. For this example the answers are supplied below. Typically the questions
##will be answered on the fly.
## Not run: 
mykml$interactivePolyStyle(id = "mystyle")
red
.5
Y
N
N

## End(Not run)
x = data.frame(cbind(c(1, 1, 1), c(45.9178, 46.3624, 46.807), c(-59.967,-60.114, -60.321)))
names(x) = c("pid", "lat", "lon")
  
mykml$addPolygon(x, name= "poly1", description = "random polygon", styleUrl = "mystyle" )
#mykml$preview()
#Notice the line disapears on close zoom, this is an error in Google Earth. This can be mitigated by adding more #points along the line.  
\end{ExampleCode}
\end{Examples}
\inputencoding{utf8}
\HeaderA{liststyles}{Print kml Styles}{liststyles}
%
\begin{Description}\relax
Allows you to quickly view your kml styles. 
\end{Description}
%
\begin{Usage}
\begin{verbatim}
yourkml$liststyles()
\end{verbatim}
\end{Usage}
%
\begin{Author}\relax
Brent Cameron\\{}
Department of Fisheries and Oceans Canada\\{}
Population Ecology Division
\end{Author}
%
\begin{Examples}
\begin{ExampleCode}

mykml = RKmlObject()
mykml$addLineStyle(styleid = "mystyle", color = "red", transparency = .5, width = 100)
mykml$addLineStyle(styleid = "yourstyle", color = "#0000ff", transparency = 1)

mykml$addPolyStyle(styleid = "mystyle", color = "red", transparency = 1, fill = 0)
mykml$addPolyStyle(styleid = "yourstyle", color = "yellow", transparency = 1, outline = 0)

mykml$liststyles()

\end{ExampleCode}
\end{Examples}
\inputencoding{utf8}
\HeaderA{preview}{Preview a kml Object}{preview}
%
\begin{Description}\relax
Allow you to quickly view your kml object. You must install Google Earth to your computer. Make sure .kml ae associated with google earth by defult
\end{Description}
%
\begin{Usage}
\begin{verbatim}
yourkml$preview()
\end{verbatim}
\end{Usage}
%
\begin{Author}\relax
Brent Cameron\\{}
Department of Fisheries and Oceans Canada\\{}
Population Ecology Division
\end{Author}
%
\begin{References}\relax
\Rhref{http://www.google.com/earth/download/ge/agree.html}{Google Earth Download}
\end{References}
%
\begin{Examples}
\begin{ExampleCode}

mykml = RKmlObject()

mykml$addFolder(fid = "EMPTY", name = "EMPTY")
x = data.frame(cbind(c(45.9178, 46.807), c(-59.967, -60.321)))
names(x) = c("lat", "lon")
mykml$addPoint(x)
#mykml$preview()

\end{ExampleCode}
\end{Examples}
\inputencoding{utf8}
\HeaderA{removeStyle}{Remove Style}{removeStyle}
%
\begin{Description}\relax
This function is used to remove a style or remove certian style types from a style. 
\end{Description}
%
\begin{Usage}
\begin{verbatim}
yourKMLobj$removeStyle(styleid) #Remove style and all types within
yourKMLobj$removeStyle(styleid, styletype) #Remove an individual style type from a style


\end{verbatim}
\end{Usage}
%
\begin{Arguments}
\begin{ldescription}

\item[\code{styleid}] 
Mandatory. The style which you would like to remove from or remove completely.  

\item[\code{styletype}] Character string or NULL. One of PolyStyle, IconStyle, LineStyle, BalloonStyle or LabelStyle. Specify which styletype to remove from this style. If no styletype provided, the whole style will be removed.(Default = NULL) 

\end{ldescription}
\end{Arguments}
%
\begin{Author}\relax
Brent Cameron \\{}
Department of Fisheries and Oceans Canada \\{}
Population Ecology Division 
\end{Author}
%
\begin{Examples}
\begin{ExampleCode}

mykml = RKmlObject()

mykml$addLineStyle(styleid = "mystyle", color = "red", transparency = .5, width = 100)
mykml$addLineStyle(styleid = "yourstyle", color = "#0000ff", transparency = 1)

mykml$addPolyStyle(styleid = "mystyle", color = "red", transparency = 1, fill = 0)
mykml$addPolyStyle(styleid = "yourstyle", color = "yellow", transparency = 1, outline = 0)

mykml$liststyles()

mykml$removeStyle(styleid = 'yourstyle')
mykml$removeStyle(styleid = 'mystyle', styletype = "PolyStyle")

mykml$liststyles()

\end{ExampleCode}
\end{Examples}
\inputencoding{utf8}
\HeaderA{saveStyle, loadStyle}{Write/load kml style Object to/from file}{saveStyle, loadStyle}
%
\begin{Description}\relax
Write a style to file for future use. The file can be read back into an existing kml object with loadStyle. 
\end{Description}
%
\begin{Usage}
\begin{verbatim}
yourkml$saveStyle(path)
yourkml$loadStyle(path)
\end{verbatim}
\end{Usage}
%
\begin{Arguments}
\begin{ldescription}
\item[\code{path}] Mandatory. The full path to where the style will be saved/loaded. 

\end{ldescription}
\end{Arguments}
%
\begin{Author}\relax
Brent Cameron\\{}
Department of Fisheries and Oceans Canada\\{}
Population Ecology Division

\end{Author}
%
\begin{Examples}
\begin{ExampleCode}

xmaskml = RKmlObject()
path = file.path(getwd(), "xmas_style")

xmaskml$addLabelStyle(styleid = "redlabel", color = "red", transparency = .2, scale = 3)
xmaskml$addLabelStyle(styleid = "greenlabel", color = "green", transparency = 1, scale = 1)
#xmaskml$saveStyle(path)


port_starboard_kml = RKmlObject()
port_starboard_kml$addLabelStyle(styleid = "yellowlabel", color = "yellow", transparency = .2, scale = 3)
#port_starboard_kml$loadStyle(path) 
port_starboard_kml$liststyles()

\end{ExampleCode}
\end{Examples}
\inputencoding{utf8}
\HeaderA{styleBuilder}{Interactive Style Builder}{styleBuilder}
%
\begin{Description}\relax
Build a complete style with this function. The builder has the option to include mouseover effects in the style. Remember the style id that you provide as you will need it to link to features that you wish to have that style
\end{Description}
%
\begin{Usage}
\begin{verbatim}
yourkmlobj$styleBuilder()
\end{verbatim}
\end{Usage}
%
\begin{Author}\relax
Brent Cameron\\{}
Department of Fisheries and Oceans Canada\\{}
Population Ecology Division
\end{Author}
%
\begin{Examples}
\begin{ExampleCode}
mykml = RKmlObject()

##ANSWER THE QUESTIONS. For this example the answers are supplied below. Typically the questions
##will be answered on the fly.
## Not run: 
mykml$styleBuilder()
ms
Y
nms
mms
Y
Hello World
green
red
Y
N
Y
purple
9
1.1
N
N
N
Y
Hello Google
red
green
Y
N
N
N
N


## End(Not run)
lat = c(44, 44.1, 44, 44.2, 44.2, 44, 44.3, 44.3, 44)
lon = c(-60, -59.9, -59.8, -60, -59.8, -60.4, -60, -59.8, -59.4)
extrude = c(0, 0, 1, 1, 1, 0, 0, 1, 0)
altitude = c(100, 80, 100, 100, 120, 500, 500, 1000, 1000)
name = c("p1","p2","p3","p4","p5","p6","p7","p8","p9")
x = cbind(lat, lon, extrude, altitude, name)
  
mykml$addPoint(x, styleUrl = "ms", description = "This is an example description applied to all points")
#mykml$preview()

\end{ExampleCode}
\end{Examples}
\inputencoding{utf8}
\HeaderA{writekml}{Write kml Object to file}{writekml}
%
\begin{Description}\relax
Write the kml object to the specified file
\end{Description}
%
\begin{Usage}
\begin{verbatim}
yourkml$writekml(path)
\end{verbatim}
\end{Usage}
%
\begin{Arguments}
\begin{ldescription}
\item[\code{path}] 
Mandatory. The full path to you kml file. Remember that the filename must end with .kml 

\end{ldescription}
\end{Arguments}
%
\begin{Author}\relax
Brent Cameron\\{}
Department of Fisheries and Oceans Canada\\{}
Population Ecology Division

\end{Author}
%
\begin{Examples}
\begin{ExampleCode}

mykml = RKmlObject()


mykml$addLabelStyle(styleid = "labelstyle1", color = "red", transparency = .2, scale = 3)
mykml$addLabelStyle(styleid = "labelstyle2", color = "green", transparency = 1, scale = 1)
 
x = data.frame(cbind(c(45.9178, 46.807), c(-59.967, -60.321), c("labelstyle1", "labelstyle2"), c("Example1", "Example2"), c("Louisbourg", "Neil's Harbour" )))
names(x) = c("lat", "lon", "styleUrl", "name", "description")
  
mykml$addPoint(x)
#mykml$writekml(path = file.path(getwd(), "kml_outputs", "example_kml.kml"))

\end{ExampleCode}
\end{Examples}
\inputencoding{utf8}
\HeaderA{writekml}{Write kml Object to file}{writekml}
%
\begin{Description}\relax
Write the kml object to the specified file
\end{Description}
%
\begin{Usage}
\begin{verbatim}
yourkml$writekml(path)
\end{verbatim}
\end{Usage}
%
\begin{Arguments}
\begin{ldescription}
\item[\code{path}] 
Mandatory. The full path to you kml file. Remember that the filename must end with .kml 

\end{ldescription}
\end{Arguments}
%
\begin{Author}\relax
Brent Cameron\\{}
Department of Fisheries and Oceans Canada\\{}
Population Ecology Division

\end{Author}
%
\begin{Examples}
\begin{ExampleCode}

mykml = RKmlObject()


mykml$addLabelStyle(styleid = "labelstyle1", color = "red", transparency = .2, scale = 3)
mykml$addLabelStyle(styleid = "labelstyle2", color = "green", transparency = 1, scale = 1)
 
x = data.frame(cbind(c(45.9178, 46.807), c(-59.967, -60.321), c("labelstyle1", "labelstyle2"), c("Example1", "Example2"), c("Louisbourg", "Neil's Harbour" )))
names(x) = c("lat", "lon", "styleUrl", "name", "description")
  
mykml$addPoint(x)
#mykml$writekml(path = file.path(getwd(), "kml_outputs", "example_kml.kml"))

\end{ExampleCode}
\end{Examples}
\printindex{}
\end{document}
